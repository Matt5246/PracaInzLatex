% !TEX program = xelatex
% !TeX encoding = UTF-8
% !TeX spellcheck = pl-PL

% Wybierz rodzaj pracy dyplomowej 
% Pick thesis type 
\documentclass[thesis=inz]{TemplateCore/Dyplom}

% \UseRawInputEncoding

% thesis=[inz|mgr|bsc|msc]
%  * inz - praca inżynierska
%  * mgr - praca magisterska
%  * bsc - bachelor thesis
%  * msc - master thesis


\usepackage[sorting=nty, bibfile=Dyplom]{TemplateCore/Dyplom} 

%- nty (Nazwisko, Tytuł, Rok):
%- nyt (Nazwisko, Rok, Tytuł):
%- none (Bez sortowania):
% - nty (Author, Title, Year):
% - nyt (Author, Year, Title):
% - none (No sorting):

%%%%%%%%%%%%%%%%%%%%%%%%%%%%%%%%%%%%%%%%%%%%%%%%%%%%%%%%%%%%%%%%%%%%%%%%%%%
% Konfiguracja - do personalizacji
% Configuration - to be personalized
%%%%%%%%%%%%%%%%%%%%%%%%%%%%%%%%%%%%%%%%%%%%%%%%%%%%%%%%%%%%%%%%%%%%%%%%%%%
\kierunek{Informatyka}
%\specjalnosc{Sieci}
\title{Aplikacja do nauki języków z napisów do filmów udostępnianych na platformie YouTube
}
\engtitle{Aplikacja do nauki języków z napisów do filmów udostępnianych na platformie YouTube
}
\album{102488}
\author{Paweł Niziołek}
\promotor{dr hab. inż. Szczepan Paszkiel}
\date{2024}
\longdate{2023-11-26}


%%%%%%%%%%%%%%%%%%%%%%%%%%%%%%%%%%%%%%%%%%%%%%%%%%%%%%%%%%%%%%%%%%%%%%%%%%%
% Streszczenie pracy i abstract.
% Summary of the Thesis and Abstract
%%%%%%%%%%%%%%%%%%%%%%%%%%%%%%%%%%%%%%%%%%%%%%%%%%%%%%%%%%%%%%%%%%%%%%%%%%%
\streszczeniepracy{
	
Praca dyplomowa skupia się na projektowaniu i realizacji szablonu do tworzenia pracy dyplomowej przy użyciu składu tekstu \LaTeX\ . W pracy zostały  przedstawione korzyści korzystania z \LaTeX{a} w porwaniu do innych edytorów tekstu skupiając się na takich rzeczach jak automatyczne formatowania, obsługa bibliograii, profesjonalny wygląd i łatwa edycja struktury dokumentu. Teoretyczna część pracy jest poświęcona przedstawieniu krótkiej historii powstawania \LaTeX{a} oraz  korzyści płynących z korzystania z \LaTeX{a} przy tworzeniu profesjonalnie wyglądających i automatycznie formatujących się prac.

W praktycznej części pracy przedstawiono stworzony szablon z przedstawieniem jego funkcji oraz możliwością  spersonalizowania  go do indywidualnych potrzeb użytkownika. Opisany zostało również jego proces instalacji i konfiguracji tak aby ułatwić korzystanie z niego innym użytkownikom. 
} % koniec streszczenia

\slowakluczowe{Skład tekstu \LaTeX, Praca Dyplomowa, Narzędzia i środowiska pracy}

\thesisabstract{
	
The thesis focuses on the design and implementation of a template for creating a thesis using \LaTeX text composition. The thesis presents the advantages of using \LaTeX{a} in abduction to other text editors focusing on such things as automatic formatting, bibliography support, professional appearance and easy editing of the document structure. The theoretical part of the work is devoted to presenting a brief history of the creation of \LaTeX{a} and the benefits of using \LaTeX{a} to create professional-looking and automatically formatting works.

The practical part of the work presents a ready-made template with a demonstration of its functions and the possibility of personalizing it to individual user needs. Its installation and configuration process is also described so as to facilitate its use by other users.
} % end of abstract

\thesiskeywords{Text Composition \LaTeX, Dissertation, Tools and working environments.}

%%%%%%%%%%%%%%%%%%%%%%%%%%%%%%%%%%%%%%%%%%%%%%%%%%%%%%%%%%%%%%%%%%%%%%%%%%%
% Tu zaczyna się dokument
%%%%%%%%%%%%%%%%%%%%%%%%%%%%%%%%%%%%%%%%%%%%%%%%%%%%%%%%%%%%%%%%%%%%%%%%%%%
\begin{document}
% Strony nagłówkowe
% Headers
\frontpages
% Właściwa treść jest w pliku Chapters/main.tex
% Real contents is in Chapters/main.tex

\chapter{Wprowadzenie}
W dzisiejszym globalnym świecie znajomość języków obcych jest nieocenioną umiejętnością. Tradycyjne metody nauki, takie jak lekcje w klasach, samouczki i aplikacje do nauki słownictwa, często są czasochłonne i nie zawsze dostosowane do indywidualnych potrzeb ucznia. Dodatkowo, oglądanie filmów i seriali w obcym języku jest powszechnie uznawane za skuteczny sposób na poprawę umiejętności językowych, ale brakuje narzędzi, które integrują te aktywności z formalnym procesem nauki. Niniejsza praca inżynierska koncentruje się na stworzeniu innowacyjnej aplikacji webowej, która połączy te dwa aspekty, oferując użytkownikom skuteczniejsze i przyjemniejsze doświadczenie edukacyjne.  

Współczesny świat, w którym żyjemy, jest coraz bardziej zglobalizowany i wymaga od nas umiejętności komunikacji w różnych językach, a przede wszystkim w języku angielskim, który stał się międzynarodowym językiem na świecie. Wraz z rozwojem technologii, nauka języków obcych stała się bardziej dostępna i atrakcyjna. . Znajomość języków obcych nie tylko otwiera drzwi do nowych możliwości zawodowych, ale także umożliwia pełniejsze zrozumienie innych kultur i poszerza horyzonty. Wraz z dynamicznym rozwojem technologii, nauka języków obcych stała się bardziej dostępna, a tradycyjne metody nauczania ewoluowały, oferując nowe, bardziej interaktywne formy edukacji. Jednym z najpopularniejszych sposobów nauki języków jest korzystanie z platform internetowych, takich jak duoLingo, gdzie użytkownicy mogą uczyć się od podstaw słów i zdań które zostały wcześniej przygotowane. Jednakże, nauka języka obcego w ten sposób ogranicza nas w kwesti wyboru czego chcielisbyśmy się dokładnie uczyć. 

Coraz więcej osób szuka alternatywnych metod nauki, które są bardziej angażujące i interaktywne. Filmy i seriale oferują naturalny kontekst, w którym używane są różne zwroty i słownictwo, co czyni je doskonałym narzędziem do nauki języka. Oglądanie treści w języku obcym nie tylko pomaga w nauce nowych słów i zwrotów, ale także w poprawie umiejętności słuchania i rozumienia języka w różnych akcentach i dialektach. 

Aplikacja ta ma umożliwić użytkownikom aktywne uczestnictwo w procesie nauki, poprzez interaktywne narzędzia i funkcje, które wspomagają naukę słownictwa i gramatyki. Wśród nich znajdują się m.in. możliwość zapisu słów z listy napisów, które są wyświetlane pod lub obok odtwarzacza video, a także możliwość dodania ich do bazy danych, aby uniknąć powtórzeń baza nie przyjmie drugiego takiego samego słowa użytkownikowi. Użytkownik będzie miał dostęp do panelu nauki, słownika wszystkich słów, możliwości logowania z różnych urządzeń obsługujących przeglądarkę, a także do różnych sposobów prezentacji danych, takich jak słownik, flashcards czy moduł do edycji napisów. 

Aplikacja ta ma również uwzględnić elementy gamifikacji, aby zachęcić użytkowników do nauki i śledzenia postępów. Na profilu użytkownika będą widoczne wszystkie nauczone słowa, a także osobna podstrona z wykresami i informacjami o postępach. Dzięki tej aplikacji, użytkownicy będą mogli efektywnie i atrakcyjnie uczyć się języka obcego, korzystając z platformy YouTube i własnych filmów z napisami z dysku własnego komputera. Napisy których użytkownik może użyć będą w różnych formatach, więc w aplikacji będzię można wybrać rodzaj pliku i przekopiować całą zawartość lub wrzucić plik w odpowiednie miejsce, napisy muszą zostać zapisane w systemie ponieważ nie ma możliwośći zapisaniu scieżki do żadnego pliku ze względów bezpieczeństwa w internecie.  

Wybór technologii do tworzenia aplikacji webowej jest kluczowy dla jej stabilności, skalowalności i wydajności. W projekcie tej aplikacji językowej zdecydowano się na framework Next.js, który oparty jest na React i oferuje wiele korzyści. Jedną z głównych zalet Next.js jest możliwość elastycznego renderowania treści, zarówno po stronie serwera (SSR), jak i klienta (CSR). Dzięki SSR, aplikacja może szybko ładować wstępnie załadowane strony, co znacząco poprawia widoczność w wyszukiwarkach (SEO - Search Engine Optimization) i przyspiesza czas ładowania, co jest szczególnie istotne dla użytkowników korzystających z platformy edukacyjnej. CSR z kolei umożliwia dynamiczne i płynne aktualizacje interfejsu bez konieczności przeładowywania całej strony, co poprawia doświadczenie użytkownika. 

Kolejną istotną zaletą Next.js jest intuicyjny i wydajny system routingu oparty na strukturze plików. Ułatwia to organizację aplikacji i nawigację po niej, co jest kluczowe dla zachowania przejrzystości i spójności struktury. Next.js oferuje również uproszczone pobieranie danych oraz wsparcie dla różnych metod stylizacji, takich jak moduły CSS i Tailwind CSS, co pozwala na tworzenie estetycznego i responsywnego interfejsu użytkownika szybciej i łatwiej. Dodatkowo, framework zapewnia wsparcie dla TypeScript, co umożliwia tworzenie bezpiecznego i stabilnego kodu przy użyciu typów które nam podkreślą jeśli będziemy próbowali błędnie używac naszych funkcji lub zmiennych. Wszystkie te cechy czynią Next.js idealnym wyborem do stworzenia nowoczesnej i wydajnej aplikacji językowej. 

Wybór bazy danych do aplikacji webowej ma ogromne znaczenie dla jej wydajności i skalowalności. W tym projekcie zdecydowano się na MongoDB Atlas, która jest objektową bazą danych typu NoSQL. MongoDB charakteryzuje się elastyczną strukturą danych, co pozwala na szybkie i efektywne przechowywanie oraz zarządzanie różnorodnymi danymi. Dzięki objektowemu modelowi danych, MongoDB doskonale nadaje się do aplikacji, które wymagają pracy z dynamicznie zmieniającymi się strukturami danych. Jest to istotne w kontekście naszej aplikacji językowej, ponieważ umożliwia łatwe przechowywanie słów i fraz w różnych formatach i językach, co jest kluczowe dla elastyczności i funkcjonalności aplikacji. 

Jedną z kluczowych zalet MongoDB jest jej skalowalność. Baza ta umożliwia łatwe skalowanie poziome, co oznacza, że możemy dodawać nowe serwery do naszego klastra bazodanowego w miarę wzrostu ilości danych i liczby użytkowników. Jest to szczególnie ważne dla aplikacji edukacyjnych, które mogą szybko rosnąć w popularność i wymagać zwiększonej mocy obliczeniowej. Dzięki temu, nasza aplikacja będzie mogła obsługiwać rosnącą liczbę użytkowników bez utraty wydajności. Dodatkowo, MongoDB Atlas oferuje wsparcie dla replikacji danych, co zwiększa niezawodność i dostępność systemu. Funkcja replikacji zapewnia, że dane są automatycznie kopiowane na wiele serwerów, co chroni przed utratą danych i zapewnia ciągłość działania aplikacji. Dzięki tym funkcjom MongoDB Atlas jest idealnym wyborem dla naszej aplikacji, zapewniając jej wydajność, skalowalność i elastyczność w zarządzaniu danymi. 

 

\chapter{Cel i zakres pracy}

Celem niniejszej pracy jest stworzenie szablonu do prezentacji i pracy dyplomowej z wykorzystaniem środowiska \LaTeX. 
Głównymi celami podczas tworzenia szablonów są usprawnienie formatowania i estetyki. 
W kontekście \LaTeX\ dotyczy gwarancji, że szablon spełnia reguły formatowania pracy naukowej równiej jest czytelny profesjonalny i estetyczny. Aspekty które należy uwzględnić, aby usprawnić formatowanie i estetykę to: 

\begin{itemize}
\item[--] marginesy i układ strony,
\item[--] czcionka i interlinia, 
\item[--] nagłówki i stopki,
\item[--] numeracja rozdziałów i sekcji,
\item[--] rozmieszczenie grafik i tabel,
\item[--] kolory i formatowanie,
\item[--] obsługa wzorów matematycznych,
\item[--] dostosowanie do polskich standardów. 
\end{itemize}
Szerzej zostanie to omówione w późniejszych rozdziałach. 
Zastosowanie wymienionych elementów umożliwi spełnienie wymogów narzuconych odgórnie przez uczelnię, ale także przedstawia się czytelnie jaki i profesjonalnie. Całość powinna być spójna i zgodna z specyfiką pracy dyplomowej.

Zautomatyzowanie struktury dokumentu w \LaTeX\ powinno być tak skonfigurowane, aby umożliwić autorowi skupieniu się na głównie na treści bez potrzeby ręcznego dostosowywania struktury. Dlatego wiele aspektów powinno być zautomatyzowane. Elementy takie jak rozdziały, sekcje, podsekcje, spis treści, biografia, numeracja stron, rozmieszczenie rysunków i tabel i nagłówki oraz style stron. Dzięki zmechanizowaniu wymienionych elementów autor może skoncentrować się na treści, a struktura dokumentu zostanie utrzymana w spójny sposób. Zautomatyzowanie układu dokumentów w \LaTeX{u} jest jednym z kluczowych aspektów, z powodu którego wielu studentów i naukowców wybiera ten system celem składania tekstu różnego typu prac.

Zarządzanie bibliografią i przypisami jest kolejnym z kluczowych elementów w tworzeniu pracy. \LaTeX\ dostarcza narzędzia takie jak \texttt{BibTex} czy \texttt{biber} które wspomagają organizację cytowań. Dzięki narzędziom \LaTeX\ zarządzanie bibliografią i przypisami staje się mniej skomplikowanie i efektywniejsze. Trzeba zwrócić uwagę, że aby \LaTeX\ poprawnie odnosił się do cytowania i tworzenia spisu bibliograficznego należy po każdym dodaniu pozycji ponownie skompilować dokument. 

Dostosowanie do wymagań uczelni polega na spełnieniu określanych standardów, w tym przypadku wymogów Politechniki Opolskiej dotyczących formatowania i struktury prac dyplomowych. Obejmuje to przede wszystkim logo politechniki, układ stron, styl bibliografii czy inne aspekty pracy.

Wspieranie różnych języków pracy szablon umożliwia pisanie pracy zarówno w języku polskim jak i w języku angielskim. Wymaga to przede wszystkim dostosowania szablonu do specyfiki obu języków. Dostosowania takich elementów jak strona tytułowa, cytowania, numeracja rozdziałów itp. wymaga zastosowania odpowiednich komend, a także spójności i precyzji w uwzględnianiu zasad gramatyki czy stylistyki.

Przyjazność dla użytkownika polega to na dostosowaniu szablonu tak żeby nie był zbyt skomplikowany pod względem pracy z nim. Powinien ułatwiać pracę szczególnie tym użytkownikom, którzy nie mieli wcześniej styczności z \LaTeX{em}. W procesie tworzenia przyjaznego szablonu w LaTeX, takie elementy jak czytelność kodu, dokumentacja i intuicyjna modyfikacja są kluczowe. Jest to jedno z głównych założeń pracy.
Zakres pracy obejmuje realizacje wyżej wymienione punkty a ponadto szablonu z udziałem użytkownika celem wyłapania niedociągnięć czy uwzględnia sugestii. 


\chapter{Opis technologii}


Struktura dokumentów \LaTeX\ przedstawiona w listingu \ref{lst:Przykład prostego dokumentu} obejmuje proste i podstawowe polecenia. Dokumenty \LaTeX\ zaczynamy od \textbf{preambuły} która znajdująca się przed właściwą treścią dokumentu. Preambuła zawiera instrukcje i ustawienia, które wpływają na cały dokument. Najczęściej zawiera takie inforamcje jak: klasa dokumentu, ustawienia strony czy pakiety. W przypadku podanego przykładu zaczynamy od wyboru klasy którym jest \texttt{article}, więcej informacji na temat wyboru klas w dokumentach \LaTeX\ można znaleźć w podrozdziale \ref{sec:klassa}. Następnym elementem są pakiety w \LaTeX\ to zestawy makr i definicji, które rozszerzają funkcjonalność języka \LaTeX\. Dzięki pakietom możesz dostosować formatowanie, dodawać nowe polecenia czy obsługiwać specjalne elementy, takie jak tabelki, grafika czy dodatkowe czcionki. Wczytujemy pakietu w \LaTeX\ za pomocą polecenie \texttt{\textbackslash usepackage{\{Nazwa\textunderscore pakietu\}}} jednak należy pamiętać że niektóre pakiety mogą wchodzić z sobą w konflikt. Kolejnym elementem przedstawionym w listingu \ref{lst:Przykład prostego dokumentu} jest makro to niestandardowe polecenia, które można definiować i używać w celu skrócenia i uproszczenia kodu. Makra pozwalają na zdefiniowanie własnych komend, które mogą być używane w dokumencie. Mogą być one przydatne, gdy chcemy wielokrotnie używać tego samego fragmentu kodu, lub gdy chcemy nadać bardziej znaczącą nazwę bardziej skomplikowanym lub długim sekwencjom poleceń. W przedstawionym przykładzie zostało stworzone makro wstawiającą hiperłącze do platformy Overleaf. Kolejnym elementem są metadane dokumentu to informacje charakteryzujące sam dokument, takie jak tytuł, autor, data itp. Ostatnim elementem w przedstawionym przykładzie jest właściwa treść dokumentu która nie jest już częścią preambuły zaczyna się od \texttt{\textbackslash begin\{document\}} a kończy na \texttt{\textbackslash end\{document\}} wszystkie elementy zawarte pomiędzy tworzą treść. Element \texttt{\textbackslash maketitle} generuje stronę tytułową na podstawie zdefiniowanych metadanych. Za pomocą komendy \texttt{\textbackslash section\{Nazwa\textunderscore sekcji\}} tworzona jest nową sekcje w naszym dokumencie.
\clearpage 
\begin{lstlisting}[caption={Przykład prostego dokumentu}, label=lst:Przykład prostego dokumentu]
\documentclass{article}

% \usepackage[utf8]{inputenc}  % Kodowanie znaków 
\usepackage[T1]{fontenc}     % Kodowanie fontu
\usepackage[polish]{babel}   % J?zyk dokumentu
\usepackage{graphicx}        % Obsługa grafiki
\usepackage{hyperref}        % Dodanie pakietu do obsługi hiperłączy

% Makro
\newcommand{\overleaflink}{\href{https://www.overleaf.com/}{Overleaf }}

\title{Przykładowy Dokument w LaTeX}
\author{Wojtek Hyl}
\date{\today}

\begin{document}
	
	\maketitle
	
	\section{Wprowadzenie}
	Tutaj możesz opisać wprowadzenie do swojego dokumentu.
	
	\section{Link do Overleaf}
	Korzystając z makra, możesz wstawić hiperłącze do Overleaf: \overleaflink
	
\end{document}

\end{lstlisting}



\overleaflink to platforma internetowa umożliwiająca pracę z dokumentami \LaTeX, która ułatwia autorom pisanie i formatowanie tekstów naukowych. Platforma oferuje również dostęp do szablonów udostępnionych przez użytkowników. Dzięki temu, że jest to platforma online nie ma potrzeby instalowania żadnych dodatkowych programów na lokalnym komputerze. Ułatwia to dostępność i pracę z dowolnego miejsca z dostępem do internetu. Możliwość pracy online umożliwia pracę w zespołach nad jednym dokumentem w czasie rzeczywistym co jest bardzo ważnym elementem. \overleaflink umożliwia integracje z platformą GitHub, co umożliwia przechowywanie dokumentów \LaTeX w systemie kontroli wersji. Ułatwia to przechowywanie wersji projektu, śledzenie zmian oraz współpracę z innymi programistami. \overleaflink oferuje również integracje z Dropboxem która umożliwia przechowywania i synchronizacje plików w chmurze. Dropbox jest bardziej intuicyjnym narzędziem do przechowywania kodów niż GitHub jednakże nie posiada kontroli wersji. Platforma oferuje bogaty zasób funkcji takich jak na przykład natychmiastowy podgląd, sprawdzanie poprawności kodu, sugestie poprawek i wiele inny przydatnych funkcjonalności. Dodatkowo strona obsługuje wiele języków, co umożliwia pisanie oraz formatowanie w wybranym języku. Bibliografia na platformie jest zintegrowana z systemem zarządzania bibliografią. Pomaga to w generowaniu bibliografii z wybranym stylem. Posiada integracje z takim portalami jak Mendeley i Zotero, co umożliwia importowanie i eksportowanie bibliografii w różnych formatach, a także pozwala na integrację z zarządzaniem referencjami. \overleaflink oferuje także bardzo bogatą i łatwo dostępną dokumentacje oraz instrukcję obsługi pakietów. Podsumowując platforma ułatwia tworzenie dokumentów \LaTeX dla osób, które nie chcą instalować żadnego dodatkowego oprogramowania swoim komputerze lokalnym i chcą mieć dostęp do swoich dokumentów w każdym miejscu. Jest również świetnym narzędziem do pracy w zespołach które pracują online nad swoimi dokumentami. Należy jednak pamiętać że takie funkcje jak integracja z GitHubem, DropBoxem, Mendeley i Zotero są wyłącznie dostępne w płatnej subskrypcji \overleaflink. Płatna wersja również zawiera wsparcie priorytetowe (ang. priority support) co umożliwia szybsze wsparcie z strony \overleaflink w razie problemów.  

\LaTeX\ umożliwia również pracę w środowisku lokalnym. Istnieje wiele narzędzi używanych do tworzenia, kompilowania i zarządzania dokumentami \LaTeX. Jedną z istotniejszych narzędzi jakie są potrzebne to edytory. Do najczęściej używanych zaliczamy TeXShop, TeXworks i TeXstudio. Edytor TeXShop umożliwiający pracę z \LaTeX{em} na systemy OS X. Posiada prosty interfejs i jest łatwy w użyciu. Natomiast jeżeli chodzi o edytory dostępne na wielu systemach możemy do nich zaliczyć TeXworks i TeXstudio. Kolejną ważna rzeczą jaka jest potrzebna do używania \LaTeX{a} na komputerach lokalnych jest jego dystrybucja. Do najczęściej używanych dystrybucji należą TeX Live i MiKTex. TeX Live jest to dystrybucja \LaTeX\ dostępna na wielu platformach obejmująca pełen zestaw pakietów i narzędzi. MikTex jest dystrybucją \LaTeX\ przeznaczoną dla systemów Windows podobnie jak Tex Live zawiera pełen zestaw pakietów i narzędzi. Ostatnią rzeczą jaka jest potrzebna do pracy z \LaTeX{em} jest kompilator. pdf\LaTeX\ jest najczęściej używanym kompilatorem umożliwiającym bezpośrednio generację plików PDF. Obsługuje nowoczesne funkcje, takie jak przykład na hiperłącza, wbudowane obrazy oraz obsługę różnych fontów. Jednym z kolejnych kompilatorów jest Xe\LaTeX. Obsługuje on fonty systemowe co umożliwia użycie innych fontów niż standardowe w \LaTeX. Obsługuje również Unicode, co pozwala na pracę w wielu językach i kodowań znaków. LuaLa\TeX\ jest edytorem umożliwiające zaawansowane operacje programistyczne używające silnika Lua\TeX\ jest rozwinięciem pdf\LaTeX. 


\chapter{Usprawnienie formatowania i estetyki}

\section{Marginesy i układ strony }\label{sec:klassa}
W szablonach \LaTeX\ ustawienia dotyczące marginesów i układu strony są kluczowe, ponieważ wpływają na końcowy wygląd dokumentu. Jednym z ważniejszych rzeczy jest odpowiedni wybór klasy dokumentu, która jest zgodna z wymaganiami. Każda klasa ma swoje specyficzne cechy i ustawienia, które można edytować do indywidualnych potrzeb. Do najczęściej używanych klas należą \texttt{article}, \texttt{report} czy \texttt{letter}. O właściwościach klas można się dowiedzieć czytając informacje na ich temat w rozdziale 8 książki ~\cite{diller2001latex}. \LaTeX\ umożliwia ustawianie układu strony do własnych potrzeb. Ustawianie marginesów jest ważnym elementem w kontekście estetyki. Do ustawiania marginesów w \LaTeX\ służy pakiet \texttt{geometry}, który umożliwia ich dokładnie ustawianie. Marginesy w ogólnym przypadku określają pustą przestrzeń między kolumną tekstu a krawędziami strony. Właściwie dostosowane wpływają na estetykę i ogólny odbiór dokumentu. W większości przypadków, np. w pracy dyplomowej ich wartość ustawia się zazwyczaj na 2,5 cm. Jeśli chodzi o \LaTeX\ wartości marginesów można ustawić za pomocą wspomnianego wyżej przedstawionego pakietu \texttt{geometry}. Domyślne ustawienia zależą jednak od wybranej klasy dokumentu, jednak bardzo łatwo można dostosować ich wartość do własnych potrzeb. 

Kolejnym elementem w kontekście formatowania i estetyki jest określenie czy dokument ma być jednostronny czy dwustronny. Ważne jest to o tyle, że dla dokumentów dwustronnych mogą występować różnice, jeśli chodzi o numeracje stron i marginesy, jeśli chodzi o strony parzyste czy nieparzyste. W układzie dwustronnym, dostosowanie numeracji stron jest istotne, aby zachować spójność i estetykę wizualną. Często stosuje się różne układy numeracji stron na stronach lewych i prawych, a także dostosowuje ich położenie, na przykład umieszczając numerację stron na środku strony. Określenie pionowej i poziomej justyfikacji wspomaga pakiet \texttt{ragged2e}. Pozwala on na elastyczniejszą justyfikację tekstu w obrębie tych kierunków. \LaTeX\ pozwala również na określenie innych parametrów takie jak rozmiar papieru, odstęp między tekstem a nagłówkiem czy też między stopką a tekstem. Ważne jest, aby dostosować te wartości do własnych preferencji czy też określonych wymagań. Odpowiednie przystosowanie tych elementów pozwala na stworzenie dokumentu o elastycznym układzie. Gwarantuje to również profesjonalny wygląd a sam dokument staje się bardziej przyjemny w odbiorze.  

\section{Czcionka i interlinia}
Elementy takie jak czcionka i interlinia wpływają drastycznie na wygląd dokumentu, ponieważ źle wybrany font tekstu czy odstęp pomiędzy wierszami utrudniają czytelność i pogarszają wygląd dokumentu. Czcionkę dostosowujemy przy użyciu pakietu \texttt{fontspace}. Jednak trzeba pamiętać, że nie wszystkie fonty są dostępne w domyślniej wersji systemu, więc używanie niestandardowych wymaga ich dodania za pomocą pakietu \texttt{fontspec}. Interlinię w \LaTeX\ możemy dostosowywać za pomocą pakietu takiego jak \texttt{setspace}. Ustawienie odstępu między wierszami na 1 będzie skutkowało tym, że odstęp będzie równy wysokości czcionki. Analogicznie ustawienie go na 2 będzie zwiększało go dwukrotnie. W stworzony przez autora szablonie została użyta czcionka \texttt{Latin Modern Roman} która jest główną krojem pisma. Natomiast  czcionka \texttt{Latin Modern Math} została zastosowana do wzorów matematycznych. 

\section{Nagłówki i stopki}
Nagłówki i stopki w \LaTeX\ są obsługiwanie przez na przykład pakiet \texttt{fancyhdr}. Pakiet umożliwia regulację górnych (nagłówki) i dolnych (stopy) elementów każdej strony. Oferuje on możliwość dostosowywania takich elementów jak na przykład linie oddzielające od treści, odległości, numerację stron i umieszczanie nazw rozdziałów. Umożliwia także ustawianie różnych wyglądów stopki i nagłówka dla stron parzystych i nieparzystych. Ważnie jest, aby dostosowywać te elementy do potrzeb własnej pracy.

\section{Numeracja rozdziałów i sekcji}
Numeracja rozdziałów i sekcji jest automatycznie prowadzona, jednakże jest możliwość dostosowania w jaki sposób te numery są formatowane czy które struktury mają być numerowane. Domyślnie każdy rozdział jest przedstawiony w formie numeru rozdziału przed jego tytułem. \LaTeX\ za pomocą pakietu \texttt{titlesec} umożliwia zmienianie formy wyświetlania. Sekcje podobnie jak rozdziały są numerowane. Automatycznie możemy kontrolować, które sekcje są numerowa za pomocą komendy \texttt{setcounter}, jednak należy pamiętać, że nienumerowane sekcje nie pojawią się w spisie treści. Pakiet \texttt{titlesec} również umożliwia kontrolę i edycje sekcji. Warto mieć na uwadze, że zbyt duża ilość numeracji poziomów może zmniejszać czytelność dokumentu i profesjonalny wygląd. 

\section{Rozmieszczenie grafik i tabel}
W wielu dokumentach grafiki są istotnym elementem przedstawienia treści. Pakiet \texttt{graphicx} jest narzędzie dodające możliwość wstawiania grafik w \LaTeX. Otoczenie  \texttt{figure} pozwala na umieszczanie grafiki, a także umożliwia również kontrole i numerację wstawionej treści w dokumencie. Umieszczone w ten sposób grafiki automatycznie są podpisywane i opisywane, a sam spis grafik jest generowany w sposób zautomatyzowany. Kolejnym istotnym elementem przedstawiania treści w \LaTeX\ są tabele. Dzięki otoczeniu \texttt{table} istnieje precyzyjna kontrola ich położenia oraz numeracja, podobnie jak w środowisku \texttt{figure}. Spis tabel jest generowany automatycznie i może być wstawiony w dowolnym miejscu w dokumencie. Istnieje wiele edytorów internetowych które ułatwiają tworzenie tabel dla osób, które nie są zaznajomione z \LaTeX{em} lub preferują wizualne interfejsy do projektowania tabel. Pozwalają one generować kod \LaTeX\ bez konieczności ręcznego wpisywania skomplikowanych komend. Rozmieszczenie grafik i tabel wymaga dobrego poznania struktury i użycia odpowiednich środowisk. Więcej na temat użycia otoczenia \texttt{figure} w podrozdziale \ref{sec:grafika} i na temat otoczenia \texttt{table} w podrozdziale \ref{sec:tabel}. Informacji na temat mechanizmu umiejscawiania obiektów ruchomych można znaleźć np. w podrozdziale 2.11 publikacji \cite{oetiker2022nie} lub podrozdziale 8.4.1 książki \cite{kucharczyk1994wprowadzenie}. Ważnym elementem jest zadbanie o odpowiednie podpisy co znacząco ułatwia nawigację po dokumencie. 

\section{Kolory i formatowanie}
\LaTeX\ jest narzędziem umożliwiającym dostosowywanie koloru tekstu do potrzeb dokumentu. Pakietem umożliwiający kontrolę nad kolorem jest \texttt{xcolor}. Dzięki temu pakietowi można dostosować kolor tekstu i tła. Obsługuje kolory podawane po nazwie, RGB , HTML, CMYK i szaro-skali. Stosowanie kolorów w dokumentach czy prezentacjach daje możliwość podkreślenia ważnych elementów, poprawia czytelność i zrozumiałość,  a w prezentacjach poprawia atrakcyjność wizualną pracy. Jednakże pamiętać należy, że zbyt intensywne czy chaotyczne kolory przeszkadzają i utrudniają odczyt treści. \LaTeX\ jak większość edytorów tekstowych daje możliwość formatowania tekstu takich jak pogrubienia, pochylenia czy podkreślenia. 

\section{Obsługa wzorów matematycznych}
\LaTeX\ to idealne narzędzie do obsługi wzorów matematycznych. Umożliwia profesjonalne i precyzyjne przedstawianie matematyki. Dwiema głównymi trybami obsługi matematyki są tryb wewnętrzny (inline) i tryb wyłączony (display). In-line umożliwia osadzenie wzorów w tekście, natomiast tryb display umieszcza wzór w osobnej linii dokumentu. Symbole matematyczne  są zapisywane w kodzie za pomocą znaków takich jak \textbf{"+"}, \textbf{"-"}, \textbf{"*"}, \textbf{"/"}. Istnieje również możliwość potęgowania za pomocą znaku \textsuperscript i pierwiastkowania za pomocą funkcji sqrt. \LaTeX\ również obsługuje funkcje i operacje matematyczne takie jak na przykład sinus, cosinus, logarytm czy suma. Obsługuje również symbole takie jak \texttt{in} (należy do) czy \texttt{neq} (różne od). Pozwala na tworzenie macierzy za pomocą środowiska \texttt{bmatrix} (macierze kwadratowe) czy \texttt{pmatric} (macierze okrągłe). W sieci dostępne są również edytory które pomagają w tworzeniu wzorów matematycznych poprzez wizualizacje podanego przez nas wzoru i przetworzenie go w komendy \LaTeX. Kompletne zestawienie symboli, dostępnych standardowo w trybie matematycznym,
można znaleźć w podrozdziale \ref{sec:Wzory} lub w podrozdziale 3.9 publikacji \cite{oetiker2022nie} lub w dodatku A książki
\cite{diller2001latex}.


\chapter{Automatyzacja struktury dokumentu}

Automatyzację struktury dokumentu można uzyskać używając odpowiednie środowiska i polecenia. Pozwalają one na sprawne i dynamiczne formatowanie różnych części takich jak rozdziały, podrozdziały itp.  Jedną z technik, która wspomaga automatyzację dokumentu \LaTeX\ są polecenia strukturalne. W \LaTeX{u}, aby automatycznie wygenerować numerowanie i spis treści dla sekcji czy też podrozdziałów itp. należy użyć odpowiednich poleceń. Między innymi są to polecenia takie jak \texttt{\textbackslash section{\{sekcja\}}}, \texttt{\textbackslash subsection{\{Podsekcja}\}} itd. Kolejnym elementem, który wspomaga automatyzację dokumentu jest zastosowanie pętli. System składu tekstu \LaTeX\ oferuje pętle takie \texttt{for}, \texttt{foreach} z pakietu \texttt{pgffor}. Pozwalają na automatyczne powtarzanie określonych operacji, co jest szczególnie użyteczne, gdy chcemy zastosować tę samą strukturę do różnych danych czy warunków. Numeracja ustawiana jest automatycznie, jeśli chodzi o sekcje, równania czy rozdziały. Jednak, aby uniknąć ręcznego ustawienia numeracji warto zastosować automatyczne etykiety i odwołania. Realizowane jest to przez użycie takich poleceń jak \texttt{\textbackslash label{\{sec:sekcja1}\}} czy \texttt{\textbackslash ref{\{sec:sekcja}\}}. W \LaTeX{u} można definiować makra, co umożliwia dynamiczne generowanie wielokrotnie używanych struktur. Makra w LaTeX to rodzaj zdefiniowanych komend, które pozwalają na zgrupowanie określonych operacji lub tekstu pod jednym identyfikatorem.  Ważnym jest również wybór konkretnej klasy dokumentu w zależności od potrzeb czy odgórnie narzuconych wymagań. Wśród sekcji klasa dokumentu można wybrać struktury takie jak np. \texttt{report} lub \texttt{book}. Klasa \texttt{report} oferuje rozdziały a \texttt{book} także części. Pakiet \texttt{titlesec} pozwala dostosować formatowanie nagłówków sekcji, a co za tym idzie jest pomocna w pewnym elementach automatyzacji takiego dokumentu. Jeśli chodzi o szablon można utworzyć go z przedefiniowanymi elementami. Szablon może być udostępniany innym użytkownikom jak również wykorzystany w innych projektach. Elementy opisane w tej części znacząco wspomagają i poprawiają automatyzację dokumentu. Pozwalają utrzymać spójność a także poprawiają pracę nad strukturą dokumentów, szczególnie gdy są one bardzo rozbudowane. 



\chapter{Zarządzanie bibliografią i przypisami}

Jest to następny element a także kluczowy aspekt w kontekście tworzenia prac naukowych czy dokumentów akademickich. W tym celu również używane są specjalne pakiety między innymi wspomniany wcześniej \texttt{BibTeX}, a także różne komendy służące do dodawania przypisów w tekście. Narzędzie to wspomaga zarządzanie bibliografia. Gromadzi dane o cytowanych pracach z różnych źródeł. Automatycznie formatuje je zgodnie z wybranym stylem. Pakiet dołączany jest do dokumentu poprzez komendę \texttt{\textbackslash usepackage{\{natbib\}}}. W celu przechowywania informacji o źródłach, które są cytowane w dokumencie należy utworzyć plik \texttt{nazwa\textunderscore bibliografi.bib}. Przykładowa struktura takiego pliku:
\begin{lstlisting}[caption={Przykłady dodawania do bibliografii}, label=lst:Przykład dodawania do Bibliografii]
%Artykul
@article{id,
  author = {Autor, A.},
  title = {Tytul artykulu},
  journal = {Czasopismo},
  year = {2000},
  pages = {123-145},
}
%Ksiazka
@book{id,
	author = {Imie Nazwisko},
	title = {Tytul ksiazki},
	publisher = {Wydawnictwo},
	year = {2005},
	address = {Miasto},
	edition = {2},
}
%Strona internetowa
@online{id,
	author = {Imie Nazwisko},
	title = {Tytul strony internetowej},
	year = {2021},
	url = {http://www.example.com},
}
\end{lstlisting}
Elementy umieszczone w pliku \texttt{nazwabibliografi.bib} można zacytować w dokumencie za pomocą polecenia \texttt{\textbackslash cite{\{id}\}}. Należy również wybrać styl bibliografii, który definiuje jak formatowane są cytowania. Zazwyczaj odbywa się to za pomocą komendy \texttt{\textbackslash bibligraphystyle}. Określenie miejsca, gdzie ma zostać umieszczona bibliografia realizowane jest przez dodanie komendy \texttt{\textbackslash bibliography}. Zarządzanie przypisami umożliwia pakiet \texttt{footnote}. Pozwala on na łatwe dodanie przypisu dolnego. \LaTeX\ pozwala na dodanie przypisów również w obszarze wzorów matematycznych. W celu oznaczenia przypisu należy użyć \texttt{footnotemark}. Dodanie tekstu do przypisu umożliwia z kolei komenda \texttt{\textbackslash footnotetext{\{treść}\}}. Przypisy mogą również być sformatowane zgodnie z odpowiednim stylem. W tym celu należy użyć pakietu \texttt{footmisc}. Zarządzanie bibliografią i przypisami wspomaga efektywne tworzenie dokumentów. Korzystanie z pakietu \texttt{BibTeX} znacząco ułatwia cytowanie prac i zarządzanie przypisami. Więcej informacji na temat tworzenia bibliografii można znaleźć na internetowej platformie edukacyjnej \cite{Vellage2017} lub w artykule \cite{Cleary2004}. 


\chapter{Wspieranie pracy w języku polskim}

Obejmuje kilka elementów, które należy mieć na uwadze realizując ten aspekt. Wpływa to przede wszystkim na poprawne formatowanie i obsługę polskich znaków diakrytycznych. Znakami diakrytycznymi są takie litery jak ą, ę, ó, ć itp. Jeśli chodzi o kodowanie znaków należy się upewnić, że dokument jest zapisywany w kodowaniu UTF-8. Zastosowanie takiego kodowania umożliwia ich poprawną obsługę. Realizowane jest to za pomocą pakietu \texttt{\textbackslash usepackage[utf8]{\{inputenc}\}}. W celu skorzystania z polskich liter należy włączyć pakiet polski lub babel z ustawieniem języka na polski. Jeśli w dokumencie używane są komendy generujące datę, to w takim przypadku trzeba dostosować ją do polskiego formatu. Zrealizować to można pakietem \texttt{\textbackslash usepackage{\{datetime}\}} \texttt{\textbackslash renewcommand{\{dateseparator}\}{\{.}\}}. W zależności od klasy dokumentu, w szczególnym przypadku \texttt{report} lub \texttt{article}, tytuły takie jak rozdział czy sekcja mogą zostać dostosowane do języka polskiego np.
\begin{lstlisting}[caption={Dostosowywanie nazwa do języka polskiego}, label=lst:Dostosowywanie nazw do języka polskiego]
\setglossarysection{section}
\makeglossaries
\loadglsentries{Glossary}
\ifnum\strcmp{\locallang}{PL} = 0 
\newcommand{\acronymstitle}{Wykaz skrótów i symboli}
\else
\newcommand{\acronymstitle}{List of Abbreviations and Symbols}
\fi
\end{lstlisting}
Jeśli w dokumencie użyte zostały przypisy należy dostosować je do konwencji polskiej. Także bibliografia powinna być dostosowana do polskich wymagań. Należy upewnić się również czy pozostałe elementy jak stopka, nagłówek czy strona tytułowa są zgodne z polskimi standardami. Wsparcie pracy w języku polskim oferuje konfigurację różnych aspektów. Przede wszystkim należy dostosować je do polskich standardów dla dokumentów akademickich czy naukowych. Odpowiednie ustawienia i dostosowania pozwalają stworzyć spójny a także profesjonalnie wyglądający dokument w języku polskim.



\chapter{Projekt szablonu pracy dyplomowej}




\section{Wprowadzenie}
Celem tego rozdziału jest przedstawienie procesu implementacji aplikacji webowej wspomagającej naukę języków obcych za pomocą filmów i seriali. W kolejnych sekcjach opisano architekturę systemu, projekt bazy danych, implementację logiki aplikacji oraz jej frontendu i backendu, zabezpieczenia przed redundancją danych oraz niepoprawnymi wpisami, możliwość logowania z różnych urządzeń obsługujących przeglądarki, implementację różnych metod nauki oraz elementów gamifikacji.
\section{interfejs użytkownika}
Interfejs użytkownika aplikacji składa się z kilku głównych widoków, które umożliwiają użytkownikom korzystanie z różnych funkcji. Wszystkie widoki zostały zaprojektowane z myślą o prostocie i intuicyjności, aby zapewnić użytkownikom łatwą nawigację i szybki dostęp do potrzebnych informacji. Poniżej przedstawiono najważniejsze elementy interfejsu użytkownika.
\subsection{Widok startowy}
Strona główna aplikacji zawiera przyciski kierujace do innych sekcji aplikacji, takich jak nauka, słownik, statystyki, profil użytkownika, ustawienia, itp. Użytkownicy mogą również przeglądać najnowsze filmy i seriale dostępne w aplikacji oraz korzystać z wyszukiwarki, aby znaleźć interesujące ich treści.
\subsection{Strona Nauki}
Strona Nauki to główne miejsce, w którym użytkownicy mogą korzystać z różnych metod nauki, takich jak fiszki, quizy, gry edukacyjne, wpisywanie tłumaczenia słowa itp.
\subsection{Strona Profilu}
\subsection{Strona Słownika}
\subsection{Strona Statystyk}
\subsection{Strona logowania oraz rejestracji}
\subsection{Strona oglądania filmów}






\chapter{Opracowanie szablonu pracy dyplomowej}







\section{Wprowadzenie}
W tej sekcji przedstawiono szczegółowy opis implementacji projektu, w tym wybór technologii, narzędzi programistycznych oraz środowiska, w którym został zrealizowany. Omówione zostaną również kluczowe aspekty techniczne, takie jak struktura bazy danych, architektura backendu i frontendu, a także proces testowania i napotkane problemy implementacyjne. Celem tej sekcji jest dostarczenie pełnego obrazu technicznego projektu oraz uzasadnienie wyboru poszczególnych rozwiązań technologicznych.


\section{Środowisko i narzędzia programistyczne}

\subsection{Środowisko programistyczne}
Do implementacji projektu wykorzystano następujące narzędzia i środowiska programistyczne:
\begin{itemize}
    \item \textbf{Visual Studio Code:} Środowisko programistyczne do tworzenia aplikacji internetowych w JavaScript i Python.
    \item \textbf{Git:} System kontroli wersji do zarządzania kodem źródłowym projektu.
    \item \textbf{MongoDB Atlas:} Usługa do hostowania bazy danych MongoDB w chmurze.
    \item \textbf{Flask:} Środowisko do tworzenia aplikacji webowych w Pythonie.
    \item \textbf{TypeScript:} Język programowania, który kompiluje się do JavaScriptu i dodaje typy danych w kodzie.
\end{itemize}
\subsection{Wybór technologii}
W trakcie realizacji projektu wykorzystano następujące technologie:
\begin{itemize}
    \item \textbf{Next.js:} Framework do tworzenia aplikacji internetowych w React, który oferuje wiele wbudowanych funkcji, takich jak routing, server-side rendering czy generowanie statyczne \cite{nextjs}.
    \item \textbf{Python:} Język programowania, który wykorzystałem do implementacji algorytmów przetwarzania języka naturalnego (NLP).
    \item \textbf{React Virtuoso:} Biblioteka do wirtualizacji list w aplikacjach internetowych.
    \item \textbf{Node.js:} Środowisko uruchomieniowe JavaScript, które pozwala na tworzenie aplikacji serwerowych.
    \item \textbf{MongoDB:} Baza danych NoSQL, która umożliwia przechowywanie danych w formacie JSON.
    \item \textbf{React.js:} Biblioteka do tworzenia interfejsów użytkownika w aplikacjach internetowych.
    \item \textbf{Shadcn:} Bibloteka gotowych komponentów do budowy interfejsu użytkownika w React.
\end{itemize}

\subsection{Opis technologii}
\subsubsection{Next.js}
Next.js to framework do tworzenia aplikacji internetowych w React, który oferuje wiele wbudowanych funkcji, takich jak routing, server-side rendering czy generowanie statyczne. Dzięki temu można tworzyć wydajne i skalowalne aplikacje internetowe, które są przyjazne dla SEO i łatwe w utrzymaniu. Next.js oferuje również wiele gotowych rozwiązań, takich jak automatyczne generowanie ścieżek, obsługa dynamicznych routów czy optymalizacja obrazów. Dzięki temu można skupić się na tworzeniu funkcjonalności, zamiast martwić się o konfigurację i optymalizację aplikacji \cite{nextjs}.

\subsubsection{Python}
Python to język programowania, który wykorzystałem do implementacji algorytmów przetwarzania języka naturalnego (NLP). Python jest popularny w dziedzinie analizy danych i uczenia maszynowego, dzięki czemu można znaleźć wiele gotowych bibliotek i narzędzi do przetwarzania tekstu. W moim projekcie wykorzystałem biblioteki takie jak NLTK, spaCy czy TextBlob do lematyzacji, oznaczania części mowy i analizy sentymentu tekstu.

\subsubsection{React Virtuoso}
React Virtuoso to biblioteka do wirtualizacji list w aplikacjach internetowych. Umożliwia renderowanie długich list danych w sposób efektywny i wydajny, co przyczynia się do poprawy wydajności i płynności interfejsu użytkownika. Dzięki React Virtuoso można renderować tylko widoczne elementy i kilka dodatkowych poza obszarem.

\subsubsection{Node.js}
Node.js to środowisko uruchomieniowe JavaScript, które pozwala na tworzenie aplikacji serwerowych. Dzięki Node.js można pisać zarówno frontend, jak i backend w jednym języku programowania, co ułatwia rozwój i utrzymanie aplikacji. Node.js oferuje również wiele gotowych modułów i bibliotek, które ułatwiają tworzenie aplikacji internetowych, takich jak Express.js, Socket.io czy Mongoose.

\subsubsection{MongoDB}
MongoDB to baza danych NoSQL, umożliwiająca przechowywanie danych w formacie JSON. Baza jest skalowalna i elastyczna, co pozwala na przechowywanie różnych typów danych i szybki dostęp do nich. W projekcie baza danych została wykorzystana do przechowywania danych o użytkownikach, słowach do nauki, napisach oraz postępach w nauce.

\subsubsection{React.js}
React.js to biblioteka do tworzenia interfejsów użytkownika w aplikacjach internetowych. React.js oferuje wiele funkcji i narzędzi, które ułatwiają tworzenie interaktywnych i responsywnych interfejsów. Dzięki React.js można tworzyć komponenty UI, zarządzać stanem aplikacji i reagować na interakcje użytkownika w sposób efektywny i wydajny.

\subsubsection{Shadcn}
Shadcn to kolekcja komponentów, które można kopiować i wklejać do swoich aplikacji. Nie jest to biblioteka komponentów, którą można zainstalować jako zależność. Shadcn nie jest dostępny ani dystrybuowany przez npm (node package manager). Użytkownik wybiera potrzebne komponenty, kopiuje i wkleja kod do swojego projektu, a następnie dostosowuje go do swoich potrzeb. Kod jest własnością użytkownika. Shadcn może służyć jako odniesienie do budowy własnych bibliotek komponentów. do rozbudowy interfejsu użytkownika w React. Shadcn oferuje wiele gotowych rozwiązań, takich jak przyciski, formularze, tabele czy karty, które można łatwo dostosować do własnych potrzeb. Dzięki Shadcn można tworzyć interfejsy użytkownika w sposób szybki i efektywny, co przyczynia się do skrócenia czasu potrzebnego na rozwój aplikacji.




\section{Baza danych}
\section{Backend}
\section{Frontend}
\section{Testy}
\section{Problemy implementacyjne}
\subsection{Wirtualizacja List}
Wirtualizacja listy w aplikacjach internetowych to technika, która optymalizuje renderowanie długich list danych. Bez niej aplikacja renderuje wszystkie elementy listy na raz, co może prowadzić do problemów z wydajnością, zwłaszcza gdy lista jest duża. Virtualizacja polega na renderowaniu jedynie tych elementów, które aktualnie są widoczne w przeglądarce użytkownika, dzięki czemu zużycie zasobów jest minimalne, a aplikacja działa płynniej. Bez wirtualizacji listy, użytkownik mógłby doświadczać opóźnień w interakcji z interfejsem, tym większych im dłuższa lista danych przy 2 tysiącach wierszy opóźnienie stawało się uciążliwe ponieważ czekało sie pare sekund na reakcje interfejsu listy i pokazanie okna dodawania trudnych słów do systemu.

\subsection*{Jak działa wirtualizacja listy}
\begin{itemize}
    \item \textbf{Obserwacja widocznych elementów:} Komponent śledzi pozycję widoku użytkownika w liście. Renderowane są tylko te elementy, które mieszczą się w aktualnie widocznym obszarze (viewport) oraz kilka dodatkowych elementów „na zapas” wokół tego obszaru.
    \item \textbf{Renderowanie na żądanie:} Gdy użytkownik przewija listę, niewidoczne elementy są dynamicznie usuwane z DOM-u, a nowe – wczytywane na ich miejsce.
    \item \textbf{Stała wysokość elementów (lub szacowana):} Dla prawidłowego działania, komponent wirtualizujący często wymaga, aby elementy listy miały stałą lub przynajmniej przewidywalną wysokość. Dzięki temu może łatwo obliczać, które elementy powinny być aktualnie wyświetlane.
    \item \textbf{Oszczędność zasobów:} Dzięki renderowaniu tylko niewielkiej liczby elementów, zmniejsza się zużycie pamięci i obciążenie procesora, co prowadzi do szybszego działania aplikacji.
\end{itemize}

\subsection*{Dlaczego React Virtuoso}
React Virtuoso jest biblioteką do wirtualizacji list, która znacznie upraszcza implementację tego mechanizmu w React. Automatycznie obsługuje:
\begin{itemize}
    \item \textbf{Przewijanie:} Zajmuje się wykrywaniem widocznych elementów, reagując na przewijanie użytkownika.
    \item \textbf{Niestandardowe wysokości elementów:} Obsługuje zarówno stałe, jak i zmienne wysokości elementów, co czyni go bardziej elastycznym.
    \item \textbf{Lazy loading:} Umożliwia ładowanie danych w locie, co jest kluczowe dla dużych list z elementami, które mogą być dynamicznie ładowane z serwera.
\end{itemize}

\subsection*{Zastosowania}
Virtualizacja listy jest szczególnie przydatna w przypadku:
\begin{itemize}
    \item \textbf{Długich list:} Kiedy lista zawiera setki lub tysiące elementów.
    \item \textbf{Aplikacji mobilnych:} Gdzie zasoby są ograniczone i każda optymalizacja wydajności jest istotna.
    \item \textbf{Interfejsów użytkownika z dużą ilością dynamicznych danych:} Takich jak portale społecznościowe, aplikacje e-commerce czy dashboardy.
\end{itemize}

\subsection*{Wady}
Mimo licznych zalet, wirtualizacja listy ma również pewne wady:
\begin{itemize}
    \item \textbf{Złożoność implementacji:} Wprowadzenie wirtualizacji może wymagać dodatkowego kodu i konfiguracji, co może zwiększyć złożoność projektu.
    \item \textbf{Problemy z dostępnością:} Renderowanie dynamiczne może wpływać na narzędzia do czytania ekranu i inne technologie wspomagające, co może utrudniać dostępność aplikacji.
\end{itemize}

Wykorzystanie React Virtuoso przyczyniło się do poprawy wydajności i płynności interfejsu użytkownika  aplikacji, co miało kluczowe znaczenie dla zadowolenia użytkowników i jakości doświadczenia użytkownika.

\subsection{Przetwarzanie języka naturalnego (NLP)}
W pracy inżynierskiej zastosowano techniki lematyzacji oraz oznaczania części mowy (POS tagging) w ramach przetwarzania języka naturalnego (NLP). Obie te techniki odegrały kluczową rolę w analizie tekstu i umożliwiły bardziej precyzyjne przetwarzanie danych z napisów, przy zapisywaniu wybranych przez użytkownika słów do nauki, lub przy wyświetlaniu częstości występowania słów w napisach \cite{NLPforNLP}.

\subsection*{Lematyzacja}
\subsection*{Lematyzacja}
Proces lematyzacji polega na sprowadzaniu różnych form gramatycznych wyrazów do ich podstawowej formy, zwanej lematem. Dzięki temu możliwe jest ujednolicenie wyrazów, które w zależności od kontekstu występują w różnych odmianach gramatycznych. Przykładowo, formy takie jak "chodzę", "chodził" czy "chodziliśmy" są sprowadzane do podstawowej formy "chodzić". Umożliwia to bardziej spójne analizowanie tekstów i wyciąganie wniosków na temat ich zawartości, np. poprzez obliczanie częstotliwości występowania poszczególnych słów \cite{NLPforNLP}.

\subsection*{POS Tagging (oznaczanie części mowy)}
Drugą techniką było oznaczanie części mowy, czyli przypisywanie każdemu słowu w tekście odpowiedniej etykiety gramatycznej (rzeczownik, czasownik, przymiotnik itd.)  \cite{NLPforNLP}. Dzięki temu możliwe było lepsze zrozumienie struktury zdań oraz funkcji słów w kontekście. Oznaczanie części mowy okazało się kluczowe w procesie analizy tekstu, umożliwiając podział słów na kategorie zależne od ich funkcji gramatycznej. W przyszłości może to być przydatne do tworzenia funkcji umożliwiających użytkownikom wybór nauki określonych kategorii słów, takich jak czasowniki czy przymiotniki itd.





\chapter{Podsumowanie}
% Projekt aplikacji webowej wspomagającej naukę języków obcych za pomocą filmów i seriali stanowi nowoczesne i proste narzędzie edukacyjne. Integracja funkcji nauki oraz oglądania treści wideo w jednym miejscu nie tylko upraszcza proces nauki, ale także eliminuje potrzebę korzystania z wielu różnych aplikacji i narzędzi. To podejście pozwala na maksymalne wykorzystanie czasu użytkownika, zwiększając efektywność nauki poprzez dostęp do materiałów w naturalnym kontekście językowym.

\section{Wnioski}

Projekt aplikacji webowej wspomagającej naukę języków obcych za pomocą filmów i seriali stanowi nowoczesne narzędzie edukacyjne. Integracja funkcji nauki i oglądania w jednym miejscu ułatwia użytkownikom proces przyswajania wiedzy, eliminując konieczność korzystania z dodatkowych aplikacji. Aplikacja umożliwia przechowywanie trudnych słów, śledzenie postępów oraz korzystanie z panelu nauki, co czyni ją wszechstronnym narzędziem wspomagającym naukę języków. Ponadto, interaktywne podejście do nauki poprzez kontekstualne użycie języka w filmach i serialach zwiększa zaangażowanie użytkowników i poprawia efektywność nauki. Dzięki temu użytkownicy mogą uczyć się w sposób bardziej naturalny i przyjemny, co sprzyja długotrwałemu zapamiętywaniu nowych informacji.

Celem pracy inżynierskiej było stworzenie aplikacji, która pozwoli użytkownikom na naukę języka w sposób bardziej naturalny, efektywny i dostosowany do indywidualnych potrzeb. Cel ten został zrealizowany poprzez zaprojektowanie systemu, który integruje naukę języka z oglądaniem treści wideo, co czyni proces edukacyjny bardziej dynamicznym i przyjaznym dla użytkownika.



\section{Kierunki dalszego rozwoju}
% Dalszy rozwój aplikacji może obejmować dodanie nowych funkcji, takich jak integracja z innymi platformami streamingowymi, rozwój panelu nauki o dodatkowe metody jak, gry edukacyjne czy interaktywne ćwiczenia. Dodatkowe modele NLP do innych języków również mogą okazać się przydatne, umożliwiając naukę mniej popularnych języków takich jak chiński gdzie problemem jest brak przerw między słowami co utrudnia aplikacji dodawanie trudnych słów do bazy danych. Ponadto, wprowadzenie funkcji społecznościowych, takich jak fora dyskusyjne czy możliwość współpracy z innymi użytkownikami, mogłoby zwiększyć zaangażowanie i motywację do nauki.

Dalszy rozwój aplikacji powinien obejmować rozbudowę jej funkcjonalności oraz poszerzenie zakresu dostępnych możliwości. W pierwszej kolejności wskazane jest wprowadzenie integracji z popularnymi platformami streamingowymi, co pozwoli na korzystanie z różnorodnych materiałów wideo bez konieczności dodatkowej konfiguracji.

Panel nauki może zostać wzbogacony o nowe metody wspierające proces edukacyjny, takie jak gry edukacyjne, interaktywne ćwiczenia czy testy wiedzy. Rozszerzenie aplikacji o obsługę większej ilości języków mogłoby zostać zrealizowane poprzez wdrożenie zaawansowanych modeli przetwarzania języka naturalnego (NLP), które umożliwią naukę mniej popularnych języków.

Wprowadzenie funkcji społecznościowych, takich jak fora dyskusyjne, grupy nauki czy możliwość współpracy między użytkownikami, stanowiłoby kolejny krok w rozwoju aplikacji. Takie rozwiązania mogłyby sprzyjać budowaniu motywacji oraz zwiększać zaangażowanie użytkowników w proces nauki.

Rozszerzenie funkcjonalności aplikacji o mechanizmy personalizacji, takie jak algorytmy rekomendujące treści dostosowane do poziomu i zainteresowań użytkownika, mogłoby znacząco wpłynąć na efektywność nauczania. Dodanie audio do treści w procesie nauki mogłoby stanowić dodatkowy element wspierający naukę, umożliwiając użytkownikom lepsze zrozumienie wymowy i intonacji w naturalnym kontekście językowym.



% Bibliografia - musi być
% Bibliography - must exist
\bibliografia

% Strony końcowe - można zakomentować, jeśli zbędne
% Additional pages - comment out if not needed

\clearpage
% Spis rysunków
\clearpage
\listoffigures
% Spis tabel
\clearpage
\listoftables

\clearpage
\lstlistoflistings


\end{document}
%%%%%%%%%%%%%%%%%%%%%%%%%%%%%%%%%%%%%%%%%%%%%%%%%%%%%%%%%%%%%%%%%%%%%%%%%%%