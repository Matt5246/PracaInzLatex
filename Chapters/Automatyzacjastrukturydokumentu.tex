
Automatyzację struktury dokumentu można uzyskać używając odpowiednie środowiska i polecenia. Pozwalają one na sprawne i dynamiczne formatowanie różnych części takich jak rozdziały, podrozdziały itp.  Jedną z technik, która wspomaga automatyzację dokumentu \LaTeX\ są polecenia strukturalne. W \LaTeX{u}, aby automatycznie wygenerować numerowanie i spis treści dla sekcji czy też podrozdziałów itp. należy użyć odpowiednich poleceń. Między innymi są to polecenia takie jak \texttt{\textbackslash section{\{sekcja\}}}, \texttt{\textbackslash subsection{\{Podsekcja}\}} itd. Kolejnym elementem, który wspomaga automatyzację dokumentu jest zastosowanie pętli. System składu tekstu \LaTeX\ oferuje pętle takie \texttt{for}, \texttt{foreach} z pakietu \texttt{pgffor}. Pozwalają na automatyczne powtarzanie określonych operacji, co jest szczególnie użyteczne, gdy chcemy zastosować tę samą strukturę do różnych danych czy warunków. Numeracja ustawiana jest automatycznie, jeśli chodzi o sekcje, równania czy rozdziały. Jednak, aby uniknąć ręcznego ustawienia numeracji warto zastosować automatyczne etykiety i odwołania. Realizowane jest to przez użycie takich poleceń jak \texttt{\textbackslash label{\{sec:sekcja1}\}} czy \texttt{\textbackslash ref{\{sec:sekcja}\}}. W \LaTeX{u} można definiować makra, co umożliwia dynamiczne generowanie wielokrotnie używanych struktur. Makra w LaTeX to rodzaj zdefiniowanych komend, które pozwalają na zgrupowanie określonych operacji lub tekstu pod jednym identyfikatorem.  Ważnym jest również wybór konkretnej klasy dokumentu w zależności od potrzeb czy odgórnie narzuconych wymagań. Wśród sekcji klasa dokumentu można wybrać struktury takie jak np. \texttt{report} lub \texttt{book}. Klasa \texttt{report} oferuje rozdziały a \texttt{book} także części. Pakiet \texttt{titlesec} pozwala dostosować formatowanie nagłówków sekcji, a co za tym idzie jest pomocna w pewnym elementach automatyzacji takiego dokumentu. Jeśli chodzi o szablon można utworzyć go z przedefiniowanymi elementami. Szablon może być udostępniany innym użytkownikom jak również wykorzystany w innych projektach. Elementy opisane w tej części znacząco wspomagają i poprawiają automatyzację dokumentu. Pozwalają utrzymać spójność a także poprawiają pracę nad strukturą dokumentów, szczególnie gdy są one bardzo rozbudowane. 

