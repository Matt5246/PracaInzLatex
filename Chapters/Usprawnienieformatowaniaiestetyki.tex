
\section{Marginesy i układ strony }\label{sec:klassa}
W szablonach \LaTeX\ ustawienia dotyczące marginesów i układu strony są kluczowe, ponieważ wpływają na końcowy wygląd dokumentu. Jednym z ważniejszych rzeczy jest odpowiedni wybór klasy dokumentu, która jest zgodna z wymaganiami. Każda klasa ma swoje specyficzne cechy i ustawienia, które można edytować do indywidualnych potrzeb. Do najczęściej używanych klas należą \texttt{article}, \texttt{report} czy \texttt{letter}. O właściwościach klas można się dowiedzieć czytając informacje na ich temat w rozdziale 8 książki ~\cite{diller2001latex}. \LaTeX\ umożliwia ustawianie układu strony do własnych potrzeb. Ustawianie marginesów jest ważnym elementem w kontekście estetyki. Do ustawiania marginesów w \LaTeX\ służy pakiet \texttt{geometry}, który umożliwia ich dokładnie ustawianie. Marginesy w ogólnym przypadku określają pustą przestrzeń między kolumną tekstu a krawędziami strony. Właściwie dostosowane wpływają na estetykę i ogólny odbiór dokumentu. W większości przypadków, np. w pracy dyplomowej ich wartość ustawia się zazwyczaj na 2,5 cm. Jeśli chodzi o \LaTeX\ wartości marginesów można ustawić za pomocą wspomnianego wyżej przedstawionego pakietu \texttt{geometry}. Domyślne ustawienia zależą jednak od wybranej klasy dokumentu, jednak bardzo łatwo można dostosować ich wartość do własnych potrzeb. 

Kolejnym elementem w kontekście formatowania i estetyki jest określenie czy dokument ma być jednostronny czy dwustronny. Ważne jest to o tyle, że dla dokumentów dwustronnych mogą występować różnice, jeśli chodzi o numeracje stron i marginesy, jeśli chodzi o strony parzyste czy nieparzyste. W układzie dwustronnym, dostosowanie numeracji stron jest istotne, aby zachować spójność i estetykę wizualną. Często stosuje się różne układy numeracji stron na stronach lewych i prawych, a także dostosowuje ich położenie, na przykład umieszczając numerację stron na środku strony. Określenie pionowej i poziomej justyfikacji wspomaga pakiet \texttt{ragged2e}. Pozwala on na elastyczniejszą justyfikację tekstu w obrębie tych kierunków. \LaTeX\ pozwala również na określenie innych parametrów takie jak rozmiar papieru, odstęp między tekstem a nagłówkiem czy też między stopką a tekstem. Ważne jest, aby dostosować te wartości do własnych preferencji czy też określonych wymagań. Odpowiednie przystosowanie tych elementów pozwala na stworzenie dokumentu o elastycznym układzie. Gwarantuje to również profesjonalny wygląd a sam dokument staje się bardziej przyjemny w odbiorze.  

\section{Czcionka i interlinia}
Elementy takie jak czcionka i interlinia wpływają drastycznie na wygląd dokumentu, ponieważ źle wybrany font tekstu czy odstęp pomiędzy wierszami utrudniają czytelność i pogarszają wygląd dokumentu. Czcionkę dostosowujemy przy użyciu pakietu \texttt{fontspace}. Jednak trzeba pamiętać, że nie wszystkie fonty są dostępne w domyślniej wersji systemu, więc używanie niestandardowych wymaga ich dodania za pomocą pakietu \texttt{fontspec}. Interlinię w \LaTeX\ możemy dostosowywać za pomocą pakietu takiego jak \texttt{setspace}. Ustawienie odstępu między wierszami na 1 będzie skutkowało tym, że odstęp będzie równy wysokości czcionki. Analogicznie ustawienie go na 2 będzie zwiększało go dwukrotnie. W stworzony przez autora szablonie została użyta czcionka \texttt{Latin Modern Roman} która jest główną krojem pisma. Natomiast  czcionka \texttt{Latin Modern Math} została zastosowana do wzorów matematycznych. 

\section{Nagłówki i stopki}
Nagłówki i stopki w \LaTeX\ są obsługiwanie przez na przykład pakiet \texttt{fancyhdr}. Pakiet umożliwia regulację górnych (nagłówki) i dolnych (stopy) elementów każdej strony. Oferuje on możliwość dostosowywania takich elementów jak na przykład linie oddzielające od treści, odległości, numerację stron i umieszczanie nazw rozdziałów. Umożliwia także ustawianie różnych wyglądów stopki i nagłówka dla stron parzystych i nieparzystych. Ważnie jest, aby dostosowywać te elementy do potrzeb własnej pracy.

\section{Numeracja rozdziałów i sekcji}
Numeracja rozdziałów i sekcji jest automatycznie prowadzona, jednakże jest możliwość dostosowania w jaki sposób te numery są formatowane czy które struktury mają być numerowane. Domyślnie każdy rozdział jest przedstawiony w formie numeru rozdziału przed jego tytułem. \LaTeX\ za pomocą pakietu \texttt{titlesec} umożliwia zmienianie formy wyświetlania. Sekcje podobnie jak rozdziały są numerowane. Automatycznie możemy kontrolować, które sekcje są numerowa za pomocą komendy \texttt{setcounter}, jednak należy pamiętać, że nienumerowane sekcje nie pojawią się w spisie treści. Pakiet \texttt{titlesec} również umożliwia kontrolę i edycje sekcji. Warto mieć na uwadze, że zbyt duża ilość numeracji poziomów może zmniejszać czytelność dokumentu i profesjonalny wygląd. 

\section{Rozmieszczenie grafik i tabel}
W wielu dokumentach grafiki są istotnym elementem przedstawienia treści. Pakiet \texttt{graphicx} jest narzędzie dodające możliwość wstawiania grafik w \LaTeX. Otoczenie  \texttt{figure} pozwala na umieszczanie grafiki, a także umożliwia również kontrole i numerację wstawionej treści w dokumencie. Umieszczone w ten sposób grafiki automatycznie są podpisywane i opisywane, a sam spis grafik jest generowany w sposób zautomatyzowany. Kolejnym istotnym elementem przedstawiania treści w \LaTeX\ są tabele. Dzięki otoczeniu \texttt{table} istnieje precyzyjna kontrola ich położenia oraz numeracja, podobnie jak w środowisku \texttt{figure}. Spis tabel jest generowany automatycznie i może być wstawiony w dowolnym miejscu w dokumencie. Istnieje wiele edytorów internetowych które ułatwiają tworzenie tabel dla osób, które nie są zaznajomione z \LaTeX{em} lub preferują wizualne interfejsy do projektowania tabel. Pozwalają one generować kod \LaTeX\ bez konieczności ręcznego wpisywania skomplikowanych komend. Rozmieszczenie grafik i tabel wymaga dobrego poznania struktury i użycia odpowiednich środowisk. Więcej na temat użycia otoczenia \texttt{figure} w podrozdziale \ref{sec:grafika} i na temat otoczenia \texttt{table} w podrozdziale \ref{sec:tabel}. Informacji na temat mechanizmu umiejscawiania obiektów ruchomych można znaleźć np. w podrozdziale 2.11 publikacji \cite{oetiker2022nie} lub podrozdziale 8.4.1 książki \cite{kucharczyk1994wprowadzenie}. Ważnym elementem jest zadbanie o odpowiednie podpisy co znacząco ułatwia nawigację po dokumencie. 

\section{Kolory i formatowanie}
\LaTeX\ jest narzędziem umożliwiającym dostosowywanie koloru tekstu do potrzeb dokumentu. Pakietem umożliwiający kontrolę nad kolorem jest \texttt{xcolor}. Dzięki temu pakietowi można dostosować kolor tekstu i tła. Obsługuje kolory podawane po nazwie, RGB , HTML, CMYK i szaro-skali. Stosowanie kolorów w dokumentach czy prezentacjach daje możliwość podkreślenia ważnych elementów, poprawia czytelność i zrozumiałość,  a w prezentacjach poprawia atrakcyjność wizualną pracy. Jednakże pamiętać należy, że zbyt intensywne czy chaotyczne kolory przeszkadzają i utrudniają odczyt treści. \LaTeX\ jak większość edytorów tekstowych daje możliwość formatowania tekstu takich jak pogrubienia, pochylenia czy podkreślenia. 

\section{Obsługa wzorów matematycznych}
\LaTeX\ to idealne narzędzie do obsługi wzorów matematycznych. Umożliwia profesjonalne i precyzyjne przedstawianie matematyki. Dwiema głównymi trybami obsługi matematyki są tryb wewnętrzny (inline) i tryb wyłączony (display). In-line umożliwia osadzenie wzorów w tekście, natomiast tryb display umieszcza wzór w osobnej linii dokumentu. Symbole matematyczne  są zapisywane w kodzie za pomocą znaków takich jak \textbf{"+"}, \textbf{"-"}, \textbf{"*"}, \textbf{"/"}. Istnieje również możliwość potęgowania za pomocą znaku \textsuperscript i pierwiastkowania za pomocą funkcji sqrt. \LaTeX\ również obsługuje funkcje i operacje matematyczne takie jak na przykład sinus, cosinus, logarytm czy suma. Obsługuje również symbole takie jak \texttt{in} (należy do) czy \texttt{neq} (różne od). Pozwala na tworzenie macierzy za pomocą środowiska \texttt{bmatrix} (macierze kwadratowe) czy \texttt{pmatric} (macierze okrągłe). W sieci dostępne są również edytory które pomagają w tworzeniu wzorów matematycznych poprzez wizualizacje podanego przez nas wzoru i przetworzenie go w komendy \LaTeX. Kompletne zestawienie symboli, dostępnych standardowo w trybie matematycznym,
można znaleźć w podrozdziale \ref{sec:Wzory} lub w podrozdziale 3.9 publikacji \cite{oetiker2022nie} lub w dodatku A książki
\cite{diller2001latex}.
