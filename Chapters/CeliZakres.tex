\section{Cel pracy}
Głównym celem niniejszej pracy jest zaprojektowanie i implementacja aplikacji webowej wspomagającej naukę języków obcych poprzez oglądanie filmów i seriali. Aplikacja ma umożliwiać użytkownikom naukę wybranego języka obcego podczas oglądania treści multimedialnych własnego wyboru. Główne założenia to łatwość w nauce, umożliwienie użytkownikom nauki języków poprzez napisy w filmach i serialach bez potrzeby tworzenia kart do nauki w innych aplikacjach, takich jak Anki, oszczędzając przy tym dużo czasu użytkownika. Połączenie nauki z rozrywką, użytkownicy będą mogli uczyć się języka podczas oglądania swoich ulubionych filmów i seriali, co czyni naukę bardziej przyjemną i efektowną. Spersonalizowane środowisko nauki, dostęp do panelu użytkownika z podsumowaniem postępów w nauce, nauka słownictwa i gramatyki bez ograniczeń poprzez fiszki, możliwość śledzenia statystyk i wykresów przedstawiających postępy w nauce, a także gamifikacja procesu nauki poprzez nagradzanie użytkowników za osiągnięcia. Aplikacja będzie motywować użytkowników do nauki poprzez wizualizację ich postępów za pomocą statystyk, wykresów i informacji widocznych na profilu użytkownika. Zapewnienie kompatybilności, aplikacja będzie dostępna z różnych przeglądarek internetowych i urządzeń. Innowacyjne rozwiązania technologiczne, wykorzystanie sztucznej inteligencji do tłumaczenia napisów i ich przetwarzanie przy użyciu NLP, oraz zastosowanie nowoczesnych technologii webowych do zapewnienia płynnego działania aplikacji.

\section{Zakres pracy}
Zakres pracy obejmuje szczegółowe opisanie i realizację następujących etapów:

\begin{itemize}
      \item \textbf{Opracowanie architektury systemu}.
      \item \textbf{Projektowanie i implementacja bazy danych}.
      \item \textbf{Tworzenie logiki aplikacji oraz jej frontendu i backendu}.
      \item \textbf{Zabezpieczenie aplikacji przed redundancją danych oraz niepoprawnymi wpisami}.
      \item \textbf{Umożliwienie użytkownikom logowania z różnych urządzeń obsługujących przeglądarki}.
      \item \textbf{Implementacja różnych metod nauki (słownik, flashcards, edycja napisów) oraz elementów gamifikacji}.
\end{itemize}
Na rynku dostępne są różne aplikacje wspomagające naukę języków obcych, takie jak Duolingo, Anki, Quizlet, Memrise, Babbel, Rosetta Stone, LingQ, FluentU, Clozemaster, itp.
