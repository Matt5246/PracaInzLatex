\section{Cel pracy}
Głównym celem niniejszej pracy jest zaprojektowanie i implementacja aplikacji webowej wspomagającej naukę języków obcych poprzez oglądanie filmów i seriali. Aplikacja ma umożliwiać użytkownikom naukę wybranego języka obcego podczas oglądania treści multimedialnych własnego wyboru. Główne założenia to:

\begin{enumerate}
    \item \textbf{Łatwość w nauce}: Umożliwienie użytkownikom nauki języków poprzez napisy w filmach i serialach
          bez potrzeby tworzenia kart do nauki w innych aplikacjach,
          takich jak Anki oszczędzając przy tym dużo czasu użytkownika.
    \item \textbf{Połączenie nauki z rozrywką}: Użytkownicy będą mogli uczyć się języka podczas oglądania swoich ulubionych filmów i seriali,
          co czyni naukę bardziej przyjemną i efektowną.
    \item \textbf{Baza danych}: Umożliwienie użytkownikom zapisywania trudnych słów do bazy danych,
          unikając duplikacji słów za pomocą blokady w bazie danych do wyciągnięcia z tego 100\% skuteczności pomagać będzie lematyzacja słów,
          czyli proces upraszczania słowa do jego formy słownikowej.
    \item \textbf{Spersonalizowane środowisko nauki}: Dostęp do panelu użytkownika z podsumowaniem postępów w nauce,
          różnorodne metody nauki słownictwa i gramatyki (fiszki, quizy, gry edukacyjne),
          możliwość śledzenia statystyk i wykresów przedstawiających postępy w nauce,
          a także gamifikacja procesu nauki poprzez nagradzanie użytkowników za osiągnięcia.
    \item \textbf{Różne sposoby nauki}: Wgląd do danych za pomocą różnych metod jak flashcards, moduł do edycji napisów, quiz, wpisywania tłumaczenia słowa itp.
    \item \textbf{Motywacja i śledzenie postępów}: Aplikacja będzie motywować użytkowników do nauki poprzez wizualizację ich postępów za pomocą statystyk, wykresów i informacji widocznych na profilu użytkownika.
    \item \textbf{Zapewnienie kompatybilności}: Aplikacja bedzie dostępna z różnych przeglądarek internetowymi i urządzeń.
    \item \textbf{Innowacyjne rozwiązania technologiczne}: Wykorzystanie sztucznej inteligencji do tłumaczenia napisów i ich przetwarzanie przy użyciu NLP,
          oraz zastosowanie nowoczesnych technologii webowych do zapewnienia płynnego działania aplikacji.
\end{enumerate}

\section{Zakres pracy}
Zakres pracy obejmuje szczegółowe opisanie i realizację następujących etapów:

\begin{itemize}
    \item \textbf{Opracowanie architektury systemu}.
    \item \textbf{Projektowanie i implementacja bazy danych}.
    \item \textbf{Tworzenie logiki aplikacji oraz jej frontendu i backendu}.
    \item \textbf{Zabezpieczenie aplikacji przed redundancją danych oraz niepoprawnymi wpisami}.
    \item \textbf{Umożliwienie użytkownikom logowania z różnych urządzeń obsługujących przeglądarki}.
    \item \textbf{Implementacja różnych metod nauki (słownik, flashcards, edycja napisów) oraz elementów gamifikacji}.
\end{itemize}
Na rynku dostępne są różne aplikacje wspomagające naukę języków obcych, takie jak Duolingo, Anki, Quizlet, Memrise, Babbel, Rosetta Stone, LingQ, FluentU, Clozemaster, itp.
