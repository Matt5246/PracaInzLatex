
Celem niniejszej pracy jest stworzenie szablonu do prezentacji i pracy dyplomowej z wykorzystaniem środowiska \LaTeX. 
Głównymi celami podczas tworzenia szablonów są usprawnienie formatowania i estetyki. 
W kontekście \LaTeX\ dotyczy gwarancji, że szablon spełnia reguły formatowania pracy naukowej równiej jest czytelny profesjonalny i estetyczny. Aspekty które należy uwzględnić, aby usprawnić formatowanie i estetykę to: 

\begin{itemize}
\item[--] marginesy i układ strony,
\item[--] czcionka i interlinia, 
\item[--] nagłówki i stopki,
\item[--] numeracja rozdziałów i sekcji,
\item[--] rozmieszczenie grafik i tabel,
\item[--] kolory i formatowanie,
\item[--] obsługa wzorów matematycznych,
\item[--] dostosowanie do polskich standardów. 
\end{itemize}
Szerzej zostanie to omówione w późniejszych rozdziałach. 
Zastosowanie wymienionych elementów umożliwi spełnienie wymogów narzuconych odgórnie przez uczelnię, ale także przedstawia się czytelnie jaki i profesjonalnie. Całość powinna być spójna i zgodna z specyfiką pracy dyplomowej.

Zautomatyzowanie struktury dokumentu w \LaTeX\ powinno być tak skonfigurowane, aby umożliwić autorowi skupieniu się na głównie na treści bez potrzeby ręcznego dostosowywania struktury. Dlatego wiele aspektów powinno być zautomatyzowane. Elementy takie jak rozdziały, sekcje, podsekcje, spis treści, biografia, numeracja stron, rozmieszczenie rysunków i tabel i nagłówki oraz style stron. Dzięki zmechanizowaniu wymienionych elementów autor może skoncentrować się na treści, a struktura dokumentu zostanie utrzymana w spójny sposób. Zautomatyzowanie układu dokumentów w \LaTeX{u} jest jednym z kluczowych aspektów, z powodu którego wielu studentów i naukowców wybiera ten system celem składania tekstu różnego typu prac.

Zarządzanie bibliografią i przypisami jest kolejnym z kluczowych elementów w tworzeniu pracy. \LaTeX\ dostarcza narzędzia takie jak \texttt{BibTex} czy \texttt{biber} które wspomagają organizację cytowań. Dzięki narzędziom \LaTeX\ zarządzanie bibliografią i przypisami staje się mniej skomplikowanie i efektywniejsze. Trzeba zwrócić uwagę, że aby \LaTeX\ poprawnie odnosił się do cytowania i tworzenia spisu bibliograficznego należy po każdym dodaniu pozycji ponownie skompilować dokument. 

Dostosowanie do wymagań uczelni polega na spełnieniu określanych standardów, w tym przypadku wymogów Politechniki Opolskiej dotyczących formatowania i struktury prac dyplomowych. Obejmuje to przede wszystkim logo politechniki, układ stron, styl bibliografii czy inne aspekty pracy.

Wspieranie różnych języków pracy szablon umożliwia pisanie pracy zarówno w języku polskim jak i w języku angielskim. Wymaga to przede wszystkim dostosowania szablonu do specyfiki obu języków. Dostosowania takich elementów jak strona tytułowa, cytowania, numeracja rozdziałów itp. wymaga zastosowania odpowiednich komend, a także spójności i precyzji w uwzględnianiu zasad gramatyki czy stylistyki.

Przyjazność dla użytkownika polega to na dostosowaniu szablonu tak żeby nie był zbyt skomplikowany pod względem pracy z nim. Powinien ułatwiać pracę szczególnie tym użytkownikom, którzy nie mieli wcześniej styczności z \LaTeX{em}. W procesie tworzenia przyjaznego szablonu w LaTeX, takie elementy jak czytelność kodu, dokumentacja i intuicyjna modyfikacja są kluczowe. Jest to jedno z głównych założeń pracy.
Zakres pracy obejmuje realizacje wyżej wymienione punkty a ponadto szablonu z udziałem użytkownika celem wyłapania niedociągnięć czy uwzględnia sugestii. 
