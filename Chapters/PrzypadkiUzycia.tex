\section{Przypadki użycia}
Przypadki użycia (ang. use cases) są techniką modelowania funkcjonalności systemu z perspektywy użytkownika. Służą do opisywania interakcji między użytkownikami (aktorami) a systemem, które prowadzą do osiągnięcia określonego celu. Każdy przypadek użycia opisuje sekwencję kroków, które użytkownik wykonuje, aby osiągnąć zamierzony rezultat.

Przypadki użycia są często wykorzystywane w procesie analizy wymagań, ponieważ pomagają zrozumieć, jakie funkcje systemu są potrzebne i jak będą one wykorzystywane przez użytkowników. Mogą być również używane do tworzenia scenariuszy testowych, które sprawdzają, czy system spełnia określone wymagania.

Podstawowe elementy przypadku użycia to:
\begin{itemize}
    \item \textbf{Aktorzy} - osoby, organizacje lub systemy, które wchodzą w interakcję z systemem.
    \item \textbf{Cel} - rezultat, który aktor chce osiągnąć.
    \item \textbf{Scenariusz} - sekwencja kroków, które prowadzą do osiągnięcia celu.
\end{itemize}

Przypadki użycia mogą być przedstawiane w formie tekstowej lub graficznej (diagramy przypadków użycia). Diagramy przypadków użycia są częścią języka UML (Unified Modeling Language) i składają się z aktorów, przypadków użycia oraz relacji między nimi.

Poniżej przedstawiono przykładowe przypadki użycia dla aplikacji webowej wspomagającej naukę języków obcych za pomocą filmów i seriali.
\subsection{Przypadek użycia: Rejestracja}
\begin{itemize}
    \item \textbf{Aktorzy:} Użytkownik
    \item \textbf{Cel:} Utworzenie nowego konta użytkownika
    \item \textbf{Scenariusz:}
          \begin{enumerate}
              \item Użytkownik otwiera stronę rejestracji.
              \item Użytkownik wprowadza adres e-mail, hasło i potwierdzenie hasła.
              \item Użytkownik klika przycisk "Zarejestruj się".
              \item System weryfikuje poprawność danych rejestracyjnych.
              \item System tworzy nowe konto użytkownika i przekierowuje go do strony głównej aplikacji.
          \end{enumerate}
\end{itemize}

\subsection{Przypadek użycia: Logowanie}
\begin{itemize}
    \item \textbf{Aktorzy:} Użytkownik
    \item \textbf{Cel:} Zalogowanie się do aplikacji
    \item \textbf{Scenariusz:}
          \begin{enumerate}
              \item Użytkownik otwiera stronę logowania.
              \item Użytkownik wybiera metodę logowania:
                    \begin{itemize}
                        \item \textbf{Logowanie za pomocą adresu e-mail i hasła:}
                              \begin{enumerate}
                                  \item Użytkownik wprowadza adres e-mail i hasło.
                                  \item Użytkownik klika przycisk "Zaloguj się".
                                  \item System weryfikuje dane logowania.
                              \end{enumerate}
                        \item \textbf{Logowanie za pomocą konta Google:}
                              \begin{enumerate}
                                  \item Użytkownik klika przycisk "Zaloguj się przez Google".
                                  \item System przekierowuje użytkownika do strony logowania Google.
                                  \item Użytkownik wprowadza dane logowania do konta Google.
                                  \item Google weryfikuje dane logowania.
                              \end{enumerate}
                    \end{itemize}
              \item System autoryzuje użytkownika i przekierowuje go do strony głównej aplikacji.
          \end{enumerate}
\end{itemize}

\subsection{Przypadek użycia: oglądanie filmów}
\begin{itemize}
    \item \textbf{Aktorzy:} Użytkownik
    \item \textbf{Cel:} Oglądanie zapisanych filmów i seriali
    \item \textbf{Scenariusz:}
          \begin{itemize}
              \item Użytkownik otwiera stronę oglądania filmów.
              \item \textbf{Użytkownik wybiera film z listy youtube:}
                    \begin{enumerate}
                        \item Użytkownik wybiera film z listy.
                        \item System odtwarza wybrany film w oknie odtwarzacza.
                        \item Użytkownik może zmieniać język napisów, przewijać film oraz dodawać słowa do bazy danych.
                    \end{enumerate}
              \item \textbf{Użytkownik wybiera film z listy:}
                    \begin{enumerate}
                        \item Użytkownik wybiera napisy z listy.
                        \item Poniżej odtwarzacza wybiera odpowiedni film z swojego dysku.
                        \item System odtwarza wybrany film w oknie odtwarzacza.
                        \item Użytkownik może zmieniać język napisów, przewijać film oraz dodawać słowa do bazy danych.
                    \end{enumerate}
          \end{itemize}
\end{itemize}

\subsection{Przypadek użycia: Nauka słówek}
\begin{itemize}
    \item \textbf{Aktorzy:} Użytkownik
    \item \textbf{Cel:} Nauka słówek za pomocą fiszek
    \item \textbf{Scenariusz:}
          \begin{enumerate}
              \item Użytkownik otwiera stronę nauki słówek (FlashCards).
              \item Użytkownik wybiera zestaw fiszek do nauki.
              \item System wyświetla fiszki z wybranego zestawu.
              \item Użytkownik przegląda fiszki i próbuje odgadnąć tłumaczenie.
              \item Użytkownik sprawdza poprawność tłumaczenia i ocenia swoją odpowiedź.
              \item System zapisuje wynik i przechodzi do kolejnej fiszki.
              \item Po zakończeniu nauki system zapisuje postępy.
          \end{enumerate}
\end{itemize}

