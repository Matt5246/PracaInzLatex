



\section{Wprowadzenie}
Celem tego rozdziału jest przedstawienie procesu implementacji aplikacji webowej wspomagającej naukę języków obcych za pomocą filmów i seriali. W kolejnych sekcjach opisano architekturę systemu, projekt bazy danych, implementację logiki aplikacji oraz jej frontendu i backendu, zabezpieczenia przed redundancją danych oraz niepoprawnymi wpisami, możliwość logowania z różnych urządzeń obsługujących przeglądarki, implementację różnych metod nauki oraz elementów gamifikacji.
\section{interfejs użytkownika}
Interfejs użytkownika aplikacji składa się z kilku głównych widoków, które umożliwiają użytkownikom korzystanie z różnych funkcji. Wszystkie widoki zostały zaprojektowane z myślą o prostocie i intuicyjności, aby zapewnić użytkownikom łatwą nawigację i szybki dostęp do potrzebnych informacji. Poniżej przedstawiono najważniejsze elementy interfejsu użytkownika.
\subsection{Widok startowy}
Strona główna aplikacji zawiera przyciski kierujace do innych sekcji aplikacji, takich jak nauka, słownik, statystyki, profil użytkownika, ustawienia, itp. Użytkownicy mogą również przeglądać najnowsze filmy i seriale dostępne w aplikacji oraz korzystać z wyszukiwarki, aby znaleźć interesujące ich treści.
\subsection{Strona Nauki}
Strona Nauki to główne miejsce, w którym użytkownicy mogą korzystać z różnych metod nauki, takich jak fiszki, quizy, gry edukacyjne, wpisywanie tłumaczenia słowa itp.
\subsection{Strona Profilu}
\subsection{Strona Słownika}
\subsection{Strona Statystyk}
\subsection{Strona logowania oraz rejestracji}
\subsection{Strona oglądania filmów}




