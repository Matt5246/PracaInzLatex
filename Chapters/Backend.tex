
\section{Backend}
Backend aplikacji został zrealizowany przy użyciu dwóch technologii: Next.js oraz Flask. Każda z tych technologii pełni inną rolę w architekturze aplikacji, co pozwala na wykorzystanie ich mocnych stron w Flask został wykonany moduł pełniący funkcje lemmatyzacji i pos taggingu ponieważ modele NLP są tam bardzo mocno rozwinięte. W Next.js zostały wykonane endpointy do komunikacji z bazą danych oraz do obsługi logiki biznesowej aplikacji. Dzięki temu możliwe jest oddzielenie części serwerowej od frontendowej, co pozwala wykorzystać zalety szybkiego serwowania stron statycznych. Poniżej przedstawiono architekturę backendu z wykorzystaniem Next.js i Flask. Dodatkowo został wykorzystany moduł LibretTranslate do tłumaczenia maszynowego w aplikacji.

\subsubsection{Next.js}
Next.js jest wykorzystywany do obsługi części serwerowej aplikacji, która jest odpowiedzialna za renderowanie stron po stronie serwera (server-side rendering) oraz generowanie statycznych stron (static site generation) \cite{bui2023next}. Dzięki temu aplikacja jest szybka i przyjazna dla SEO. Next.js umożliwia również łatwe zarządzanie ścieżkami strony które są wyświetlane w pasku url przeglądarki. Głównym celem w projekcie jest obsługa żądań HTTP, komunikacja z bazą danych MongoDB oraz wykonywanie operacji związanych z komunikacją z modułami NLP w Flask jak i modułem LibretTranslate do tłumaczenia maszynowego w aplikacji. Next.js oferuje również wsparcie dla różnych metod autoryzacji i uwierzytelniania, co zapewnia bezpieczeństwo aplikacji jak i szybkość działania i implementacji \cite{bugl2024modern}.
\subsection{Struktura backendu w Next.js}
Struktura backendu w Next.js została podzielona na kilka głównych katalogów, z których każdy odpowiada za określoną część aplikacji. Wszystkie pliki związane z budową backendu znajdują się w katalogu \texttt{src/app/api}, który zawiera następujące podkatalogi:

\begin{itemize}
    \item \textbf{auth:} Obsługuje autoryzację i uwierzytelnianie użytkowników.
    \item \textbf{captions:} Zajmuje się pobieraniem napisów z youtube.
    \item \textbf{hardWords:} Zajmuje się zarządzaniem trudnymi słowami.
    \item \textbf{profile:} Obsługuje aktualizacje profilu użytkownika.
    \item \textbf{sentence:} Zajmuje się aktualizacjami zdań.
    \item \textbf{signup:} Obsługuje rejestrację nowych użytkowników.
    \item \textbf{subtitles:} Zajmuje się zarządzaniem napisami do filmów.
\end{itemize}

\subsection{Struktura backendu w Flask}
Flask jest wykorzystywany do tworzenia API oraz logiki biznesowej aplikacji. Flask to lekki framework webowy dla Pythona, który umożliwia szybkie tworzenie i rozwijanie aplikacji webowych. W projekcie Flask jest odpowiedzialny za przetwarzanie żądań HTTP, oraz wykonywanie operacji związanych z przetwarzaniem języka naturalnego (NLP) \cite{fountoulakis2024evaluating} \cite{abusahyon2023ai}.
użyte do tego zostały modele:

\begin{lstlisting}[language=Python, caption=kod do importowania modeli NLP]
    nlp_de = spacy.load('de_core_news_md')
    nlp_ja = spacy.load('ja_core_news_md')
    nlp_en = spacy.load('en_core_web_md')
    nlp_pl = spacy.load('pl_core_news_md')
    
\end{lstlisting}

zostały one użyte do przetwarzania języka naturalnego w aplikacji. Dzięki temu możliwe jest lemmatyzacja i pos tagging słów w różnych językach. Do projektu zostały wykorzystane modele językowe dla języków: niemieckiego, japońskiego, angielskiego i polskiego. Reszta języków nie jest obsługiwana w aplikacji, ale można dodać nowe modele językowe w przyszłości.

\begin{lstlisting}[language=Python, caption=kod do obsługi lemmatyzacji i pos taggingu]
    @app.post("/nlp")
    async def analyze_text(request: AnalyzeTextRequest):
        word = request.word
        sourceLang = request.sourceLang
        doc = None
    
        if not sourceLang:
            raise HTTPException(status_code=400, detail="Source language is required")
        if sourceLang == 'de':
            doc = nlp_de(word)
        elif sourceLang == 'ja':
            doc = nlp_ja(word)
        elif sourceLang == 'en':
            doc = nlp_en(word)
        elif sourceLang == 'pl':
            doc = nlp_pl(word)
        elif sourceLang == 'auto':
            doc = nlp_de(word)
        else:
            raise HTTPException(status_code=400, detail=f"Source language is not currently used: {sourceLang}")
        
        tokens_with_pos = {'lemma': doc[0].lemma_, 'pos': doc[0].pos_}
        return {'result': tokens_with_pos}
\end{lstlisting}

Funkcja \texttt{analyze\_text} przyjmuje żądanie zawierające słowo oraz język źródłowy,
a następnie przetwarza tekst przy użyciu odpowiedniego modelu językowego. W przypadku braku określenia języka źródłowego lub podania nieobsługiwanego języka, zwracany jest odpowiedni komunikat błędu.
Wynikiem przetwarzania jest lemat naszego słowa oraz oznaczenie części mowy w obu przypadkach wybrane są te najbardziej prawdopobone w przetworzonym dokumencie który zwrócił model NLP.


\subsection{LibretTranslate}
LibretTranslate to otwartoźródłowy system tłumaczenia maszynowego, który umożliwia tłumaczenie tekstów pomiędzy różnymi językami. W projekcie został wykorzystany do tłumaczenia tekstów w aplikacji, co pozwala na obsługę użytkowników mówiących różnymi językami. LibretTranslate wspiera wiele języków, co czyni go wszechstronnym narzędziem do tłumaczeń.

\subsubsection{Integracja LibretTranslate}
Integracja LibretTranslate w aplikacji została zrealizowana poprzez stworzenie funkcji w Next.js, która komunikuje się z API LibretTranslate. Dzięki temu możliwe jest wysyłanie napisów do przetłumaczenia oraz odbieranie przetłumaczonych wyników bezpośrednio w aplikacji. Poniżej przedstawiono przykładowy kod obsługujący tłumaczenie napisów za pomocą LibretTranslate:

\begin{lstlisting}[language=JavaScript, caption=Przykładowy kod integracji LibretTranslate w Next.js]
import axios from 'axios';
export default async function translateText(req, res) {
    async function translateSubtitleData(subtitleData: SubtitleData[], targetLang: string) {
    try {
        const texts = subtitleData.map(subtitle => subtitle?.text);
        const response = await axios.post("http://127.0.0.1:5000/translate", {
            q: texts,
            source: "auto" ,
            target: targetLang || "en",
            format: "text"
        });
        let detectedLanguage = "auto";
        if(response.data.detectedLanguage[0].language){
            detectedLanguage = response.data.detectedLanguage[0].language;
        }
        return {
            translatedSubtitleData: response.data.translatedText,
            detectedLanguage
        };
    } catch (error) {
        console.error('Error translating subtitles:', error);
        throw new Error('Failed to translate subtitles');
    }
}
}
\end{lstlisting}

Funkcja \texttt{translateSubtitleData} przyjmuje dane napisów oraz język docelowy, a następnie wysyła zapytanie do API LibretTranslate w celu przetłumaczenia tekstu. Odpowiedź zawiera przetłumaczone napisy oraz wykryty język źródłowy, co pozwala na dalsze przetwarzanie danych w aplikacji.

\subsection{Przetwarzanie języka naturalnego (NLP)}
W pracy inżynierskiej zastosowano techniki lematyzacji oraz oznaczania części mowy (POS tagging) w ramach przetwarzania języka naturalnego (NLP). Obie te techniki odegrały kluczową rolę w analizie tekstu i umożliwiły bardziej precyzyjne przetwarzanie danych z napisów, przy zapisywaniu wybranych przez użytkownika słów do nauki, lub przy wyświetlaniu częstości występowania słów w napisach \cite{NLPforNLP}.

\subsection*{Lematyzacja}
Proces lematyzacji polega na sprowadzaniu różnych form gramatycznych wyrazów do ich podstawowej formy, zwanej lematem. Dzięki temu możliwe jest ujednolicenie wyrazów, które w zależności od kontekstu występują w różnych odmianach gramatycznych. Przykładowo, formy takie jak "chodzę", "chodził" czy "chodziliśmy" są sprowadzane do podstawowej formy "chodzić". Umożliwia to bardziej spójne analizowanie tekstów i wyciąganie wniosków na temat ich zawartości, np. poprzez obliczanie częstotliwości występowania poszczególnych słów \cite{NLPforNLP}.

\subsection*{POS Tagging (oznaczanie części mowy)}
Drugą techniką było oznaczanie części mowy, czyli przypisywanie każdemu słowu w tekście odpowiedniej etykiety gramatycznej (rzeczownik, czasownik, przymiotnik itd.)  \cite{NLPforNLP}. Dzięki temu możliwe było lepsze zrozumienie struktury zdań oraz funkcji słów w kontekście. Oznaczanie części mowy okazało się kluczowe w procesie analizy tekstu, umożliwiając podział słów na kategorie zależne od ich funkcji gramatycznej. W przyszłości może to być przydatne do tworzenia funkcji umożliwiających użytkownikom wybór nauki określonych kategorii słów, takich jak czasowniki czy przymiotniki itd.



