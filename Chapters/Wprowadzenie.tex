
W prezentacjach profesjonalnych dokumentów ważną rolę odgrywa skład tekstu. W zakresie dostępnych rozwiązań, \LaTeX\ pokazał się jakie mocne narzędzie pozwalające tworzyć teksty wysokiej jakości, zwłaszcza jeśli chodzi o artykuły książki i prace naukowe. Praca ma na celu stworzenie szablonu \LaTeX\ do pracy dyplomowej i prezentacji. W następnych rozdziałach omówione zostaną informacje teoretyczne dotyczące formatowania dokumentów \LaTeX\ . W rozdziale 8 zostało omówione sposób korzystania z szablonu przygotowanego przez autora pracy.
 
\LaTeX\ jest systemem składu tekstu opartym na \TeX\ stworzonym przez Donalda E. Knutha \cite{knuth1996texbook}, a sam został napisany przez Lesliego Lamporta w latach 80 \cite{lamport1985latex}. XX wieku. \TeX\  został opracowany jako system składu tekstu przez Donalda Knutha w latach 70. XX wieku, który chciał stworzyć narzędzie umożliwiające precyzyjne i estetyczne składanie tekstu, przede wszystkim w dziedzinie matematyki i nauk komputerowych. W roku 1983 została opublikowana pierwsza wersja \LaTeX\ przez Leslie Lamporta, stanowiąca zestaw makr na bazie \TeX{a}. Jednak \TeX\ był bardzo skomplikowanym narzędziem dla osób nieznających go, a \LaTeX\ miał uprościć i ułatwić pracę dokumentami. Zaczął zyskiwać popularność wśród naukowców, inżynierów, matematyków oraz osób potrzebujących profesjonalnego narzędzia do składania tekstu w dziedzinach naukowych. \LaTeX\ był  otwatym oprogramowaniem (ang. open-source), co wpłynęło na rozwój społeczności użytkowników. Dzięki szerszej współpracy pojawiły się pakiety, klasy dokumentów oraz szablony, które zwiększyły funkcjonalność samego \LaTeX{a}. W następstwie pojawiły się różne rozszerzenia \LaTeX{a}, takie jak na przykład Xe\LaTeX\ czy Lua\LaTeX, które umożliwiły nowe możliwości, obsługę formatów czcionek oraz systemów znaków. \LaTeX\ to dynamiczny system składu tekstu, a jego różne dystrybucje, takie jak \TeX\ Live czy MiKTeX, są regularnie aktualizowane i rozwijane. Społeczność użytkowników aktywnie pracuje nad nowymi funkcjami i poprawkami. Szablony \LaTeX\ są stosowane wszędzie tam, gdzie wymagana jest precyzja i wysoka jakość składu tekstu. \LaTeX\ pozostaje popularnym narzędziem składu tekstu, ponieważ jest stabilnym i elastyczny. Stale ewoluuje, dostosowując się do zmieniających się potrzeb użytkowników oraz nowych standardów w składzie tekstu.

Zalety wynikające z korzystania z narzędzia \LaTeX\ są ogromne, dzięki czemu jest szeroko uznawanym narzędziem w środowiskach naukowych. Oferuje wysoką jakość składu tekstu, umożliwia elastyczny i profesjonalny wygląd dokumentu. Dla prac naukowych, artykułów i innych dokumentów o charakterze naukowym, \LaTeX\ stanowi kluczowy element. Zaawansowana topologia systemu pozwala na dokładne formatowanie tekstu, tabel czy równań matematycznych. Dzięki złożonej strukturze \LaTeX\ pozwala na tworzenie własnych makr i pakietów, a także skomplikowanych struktur dokumentu takie jak nagłówki, pod nagłówki, sekcje, podsekcje, gdzie autorzy mają kontrolę nad formatowaniem każdego poziomu struktury. Posiada również fachowe narzędzie do zarządzania cytowaniami i bibliografiami, gdzie system automatycznie generuję bibliografię zgodnie z wybranym stylem. Umożliwia wydzielenie zawartości od formy znaczy to, że autorzy mogą się skupić na samym tekście podczas gdy system sam zajmie się formatowaniem. Zapewnia stabilną i niezawodną pracę na różnych systemach operacyjnych i jest dostępny w formie otwartego oprogramowania (ang. open-source), co pozwala na rozwój i wsparcie od strony społeczności \LaTeX. Dzięki rozwoju technologii jest także możliwość korzystania z systemu \LaTeX\ poprzez internetowe edytor jakie jak na przykład \href{https://www.overleaf.com/}{Overleaf}
. Automatyczna numeracja spisu treści czy indeksów umożliwia pracę z dużymi dokumentami taki jak na przykład książki, rozprawy doktorskie. Autorzy nie muszą się martwić detalami formatowania, ponieważ \LaTeX\ zajmuje się nimi. \LaTeX\ ułatwia pracę w zespole nad jednym dokumentem, co jest często wykorzystywane w badaniach naukowych. Zmiany są dokonywane w czytelny sposób, dlatego ułatwia to śledzenie i zrozumiałość. Narzędzie wspiera wiele języków i umożliwia korzystanie z różnych znaków specjalnych czy akcentów, co pozwala na redagowanie tekstów które używają znaków diakrytycznych takich jak (ą, ę. ć) . Obsługuje również rożne standardy kodowań znaków jakich jak na przykład UTF-8 Latina-1 czy CP1250 .Te wyżej wymienione zalety sprawiają, że \LaTeX\ jest idealnym narzędziem dla osób wymagających precyzji i profesjonalizmu w swoich pracach. 

Pomimo tego że \LaTeX\ ma wiele zalet, istnieją również pewne wady, które mogą sprawić, że nie będzie odpowiedni dla wszystkich użytkowników.
\LaTeX\ ma duży próg wejścia dla nowych użytkowników, osoby które nie mały wcześniej kontaktu z \LaTeX\ będą potrzebowały czasu aby nauczyć się podstawowych poleceń albo struktury dokumentu. Obsługiwany jest często poprzez pisanie kodu źródłowego w edytorze. Dlatego osoby przyzwyczajone do pracy z programami oferującymi graficzny interfejs użytkownika, taką formę pracy mogą uznać za mniej intuicyjną. Dla krótkich dokumentów tekstowych takich jak listy czy notatki, \LaTeX\ może być uważany za czasochłonny i zbyt rozbudowany w porównaniu do innych narzędzi do edycji tekstu.
\LaTeX\ jest zależny od  od zewnętrznych narzędzi ponieważ wymaga kompilacji co oznacza użytkownik musi korzystać z zewnętrznych narzędzi do przetwarzania kodu na gotowy dokument PDF lub inny format. Warto jednak zauważyć że te wady mogą być znoszone poprzez doświadczenie użytkownika a niektóre mogą być łagodzone poprzez korzystanie z odpowiednich edytorów ułatwiającymi prace.
Podsumowując jest doskonałym narzędziem do tworzenia skomplikowanych dokumentów technicznych, naukowych czy akademickich. Podczas gdy tradycyjne edytory takie jak Microsoft Word czy Google Docs są bardziej dostosowane do prostszych zastosowań i oferują łatwiejszą obsługę dla osób bez fachowej wiedzy.