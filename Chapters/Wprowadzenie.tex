W dzisiejszym globalnym świecie znajomość języków obcych jest nieocenioną umiejętnością. Tradycyjne metody nauki, takie jak lekcje w klasach, samouczki i aplikacje do nauki słownictwa, często są czasochłonne i nie zawsze dostosowane do indywidualnych potrzeb ucznia. Dodatkowo, oglądanie filmów i seriali w obcym języku jest powszechnie uznawane za skuteczny sposób na poprawę umiejętności językowych, ale brakuje narzędzi, które integrują te aktywności z formalnym procesem nauki. Niniejsza praca inżynierska koncentruje się na stworzeniu innowacyjnej aplikacji webowej, która połączy te dwa aspekty, oferując użytkownikom skuteczniejsze i przyjemniejsze doświadczenie edukacyjne.  

Współczesny świat, w którym żyjemy, jest coraz bardziej zglobalizowany i wymaga od nas umiejętności komunikacji w różnych językach, a przede wszystkim w języku angielskim, który stał się międzynarodowym językiem na świecie. Wraz z rozwojem technologii, nauka języków obcych stała się bardziej dostępna i atrakcyjna. . Znajomość języków obcych nie tylko otwiera drzwi do nowych możliwości zawodowych, ale także umożliwia pełniejsze zrozumienie innych kultur i poszerza horyzonty. Wraz z dynamicznym rozwojem technologii, nauka języków obcych stała się bardziej dostępna, a tradycyjne metody nauczania ewoluowały, oferując nowe, bardziej interaktywne formy edukacji. Jednym z najpopularniejszych sposobów nauki języków jest korzystanie z platform internetowych, takich jak duoLingo, gdzie użytkownicy mogą uczyć się od podstaw słów i zdań które zostały wcześniej przygotowane. Jednakże, nauka języka obcego w ten sposób ogranicza nas w kwesti wyboru czego chcielisbyśmy się dokładnie uczyć. 

Coraz więcej osób szuka alternatywnych metod nauki, które są bardziej angażujące i interaktywne. Filmy i seriale oferują naturalny kontekst, w którym używane są różne zwroty i słownictwo, co czyni je doskonałym narzędziem do nauki języka. Oglądanie treści w języku obcym nie tylko pomaga w nauce nowych słów i zwrotów, ale także w poprawie umiejętności słuchania i rozumienia języka w różnych akcentach i dialektach. 

Aplikacja ta ma umożliwić użytkownikom aktywne uczestnictwo w procesie nauki, poprzez interaktywne narzędzia i funkcje, które wspomagają naukę słownictwa i gramatyki. Wśród nich znajdują się m.in. możliwość zapisu słów z listy napisów, które są wyświetlane pod lub obok odtwarzacza video, a także możliwość dodania ich do bazy danych, aby uniknąć powtórzeń baza nie przyjmie drugiego takiego samego słowa użytkownikowi. Użytkownik będzie miał dostęp do panelu nauki, słownika wszystkich słów, możliwości logowania z różnych urządzeń obsługujących przeglądarkę, a także do różnych sposobów prezentacji danych, takich jak słownik, flashcards czy moduł do edycji napisów. 

Aplikacja ta ma również uwzględnić elementy gamifikacji, aby zachęcić użytkowników do nauki i śledzenia postępów. Na profilu użytkownika będą widoczne wszystkie nauczone słowa, a także osobna podstrona z wykresami i informacjami o postępach. Dzięki tej aplikacji, użytkownicy będą mogli efektywnie i atrakcyjnie uczyć się języka obcego, korzystając z platformy YouTube i własnych filmów z napisami z dysku własnego komputera. Napisy których użytkownik może użyć będą w różnych formatach, więc w aplikacji będzię można wybrać rodzaj pliku i przekopiować całą zawartość lub wrzucić plik w odpowiednie miejsce, napisy muszą zostać zapisane w systemie ponieważ nie ma możliwośći zapisaniu scieżki do żadnego pliku ze względów bezpieczeństwa w internecie.  

Wybór technologii do tworzenia aplikacji webowej jest kluczowy dla jej stabilności, skalowalności i wydajności. W projekcie tej aplikacji językowej zdecydowano się na framework Next.js, który oparty jest na React i oferuje wiele korzyści. Jedną z głównych zalet Next.js jest możliwość elastycznego renderowania treści, zarówno po stronie serwera (SSR), jak i klienta (CSR). Dzięki SSR, aplikacja może szybko ładować wstępnie załadowane strony, co znacząco poprawia widoczność w wyszukiwarkach (SEO - Search Engine Optimization) i przyspiesza czas ładowania, co jest szczególnie istotne dla użytkowników korzystających z platformy edukacyjnej. CSR z kolei umożliwia dynamiczne i płynne aktualizacje interfejsu bez konieczności przeładowywania całej strony, co poprawia doświadczenie użytkownika. 

Kolejną istotną zaletą Next.js jest intuicyjny i wydajny system routingu oparty na strukturze plików. Ułatwia to organizację aplikacji i nawigację po niej, co jest kluczowe dla zachowania przejrzystości i spójności struktury. Next.js oferuje również uproszczone pobieranie danych oraz wsparcie dla różnych metod stylizacji, takich jak moduły CSS i Tailwind CSS, co pozwala na tworzenie estetycznego i responsywnego interfejsu użytkownika szybciej i łatwiej. Dodatkowo, framework zapewnia wsparcie dla TypeScript, co umożliwia tworzenie bezpiecznego i stabilnego kodu przy użyciu typów które nam podkreślą jeśli będziemy próbowali błędnie używac naszych funkcji lub zmiennych. Wszystkie te cechy czynią Next.js idealnym wyborem do stworzenia nowoczesnej i wydajnej aplikacji językowej. 

Wybór bazy danych do aplikacji webowej ma ogromne znaczenie dla jej wydajności i skalowalności. W tym projekcie zdecydowano się na MongoDB Atlas, która jest objektową bazą danych typu NoSQL. MongoDB charakteryzuje się elastyczną strukturą danych, co pozwala na szybkie i efektywne przechowywanie oraz zarządzanie różnorodnymi danymi. Dzięki objektowemu modelowi danych, MongoDB doskonale nadaje się do aplikacji, które wymagają pracy z dynamicznie zmieniającymi się strukturami danych. Jest to istotne w kontekście naszej aplikacji językowej, ponieważ umożliwia łatwe przechowywanie słów i fraz w różnych formatach i językach, co jest kluczowe dla elastyczności i funkcjonalności aplikacji. 

Jedną z kluczowych zalet MongoDB jest jej skalowalność. Baza ta umożliwia łatwe skalowanie poziome, co oznacza, że możemy dodawać nowe serwery do naszego klastra bazodanowego w miarę wzrostu ilości danych i liczby użytkowników. Jest to szczególnie ważne dla aplikacji edukacyjnych, które mogą szybko rosnąć w popularność i wymagać zwiększonej mocy obliczeniowej. Dzięki temu, nasza aplikacja będzie mogła obsługiwać rosnącą liczbę użytkowników bez utraty wydajności. Dodatkowo, MongoDB Atlas oferuje wsparcie dla replikacji danych, co zwiększa niezawodność i dostępność systemu. Funkcja replikacji zapewnia, że dane są automatycznie kopiowane na wiele serwerów, co chroni przed utratą danych i zapewnia ciągłość działania aplikacji. Dzięki tym funkcjom MongoDB Atlas jest idealnym wyborem dla naszej aplikacji, zapewniając jej wydajność, skalowalność i elastyczność w zarządzaniu danymi. 

 