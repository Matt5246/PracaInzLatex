

W celach realizacji szablonu do pracy dyplomowej i prezentacji podjęto następujące kroki :
\begin{itemize}

\item[--] przeprowadzona analizę wymagań dotyczących pracy dyplomowej uwzględniając w niej wytyczne uczelni. Zwarto wszystkie elementy które powinny znajdować się w szablonie.
\item[--] dostosowanie szablonu do oficjalnych wymagań co do  formatowania, struktury i prezentacji pracy dyplomowej.
\item[--] wybrano odpowiednią klasę dokumentu \LaTeX\ tak aby najlepiej dopasowała się do wymagań pracy dyplomowej pod względem formatowania i struktury dokumentu.
\item[--] sporządzono strukturę dokumentu uwzględniając kolejność rozdziałów, miejsce spisu treści, bibliografi i innych elementów pracy .
\item[--] dostosowanie style tekstu, ustalając fonty, marginesy, interlinię, oraz inne parametry. formatowania tak aby były zgodne z wymaganiami uczelni a zarazem tworzyły spójny wygląd 
\item[--] uruchomienia wsparcia dla elementów graficznych takich  jak rysunki, tabele czy schematy Zapewniono aby elementy były zgodne z stylem pracy.
\item[--] zaimplementowano nagłówki i stopki żeby zawierały niezbędne informacje takie jak nazwa i numer rozdziału, numeracja stron, itp.
\item[--] implementacja przypisów umożliwiająca dodawanie przypisów Górnych i dolnych 
\item[--] przeprowadzono test szablonu żeby sprawdzić jego poprawność formatowania i zgodność z wytycznymi uczelni.
\item[--] przygotowano przewodnik dla użytkownika szablonu zawierający takie informacje jak dotyczące obsługi, dostosowań, itp.
\end{itemize}

Podjęcie tych kroków pozwoliło na stworzenie profesjonalnie wyglądającego szablonu spełniającego wymagania uczelni oraz polskiej tradycji piśmienniczej.Ułatwiło również Studentom proces tworzenia profesjonalnych dokumentów dyplomowych. 