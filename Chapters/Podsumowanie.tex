% Projekt aplikacji webowej wspomagającej naukę języków obcych za pomocą filmów i seriali stanowi nowoczesne i proste narzędzie edukacyjne. Integracja funkcji nauki oraz oglądania treści wideo w jednym miejscu nie tylko upraszcza proces nauki, ale także eliminuje potrzebę korzystania z wielu różnych aplikacji i narzędzi. To podejście pozwala na maksymalne wykorzystanie czasu użytkownika, zwiększając efektywność nauki poprzez dostęp do materiałów w naturalnym kontekście językowym.

\section{Wnioski}

Projekt aplikacji webowej wspomagającej naukę języków obcych za pomocą filmów i seriali stanowi nowoczesne narzędzie edukacyjne. Integracja funkcji nauki i oglądania w jednym miejscu ułatwia użytkownikom proces przyswajania wiedzy, eliminując konieczność korzystania z dodatkowych aplikacji. Aplikacja umożliwia przechowywanie trudnych słów, śledzenie postępów oraz korzystanie z panelu nauki, co czyni ją wszechstronnym narzędziem wspomagającym naukę języków. Ponadto, interaktywne podejście do nauki poprzez kontekstualne użycie języka w filmach i serialach zwiększa zaangażowanie użytkowników i poprawia efektywność nauki. Dzięki temu użytkownicy mogą uczyć się w sposób bardziej naturalny i przyjemny, co sprzyja długotrwałemu zapamiętywaniu nowych informacji.

Celem pracy inżynierskiej było stworzenie aplikacji, która pozwoli użytkownikom na naukę języka w sposób bardziej naturalny, efektywny i dostosowany do indywidualnych potrzeb. Cel ten został zrealizowany poprzez zaprojektowanie systemu, który integruje naukę języka z oglądaniem treści wideo, co czyni proces edukacyjny bardziej dynamicznym i przyjaznym dla użytkownika.



\section{Kierunki dalszego rozwoju}
% Dalszy rozwój aplikacji może obejmować dodanie nowych funkcji, takich jak integracja z innymi platformami streamingowymi, rozwój panelu nauki o dodatkowe metody jak, gry edukacyjne czy interaktywne ćwiczenia. Dodatkowe modele NLP do innych języków również mogą okazać się przydatne, umożliwiając naukę mniej popularnych języków takich jak chiński gdzie problemem jest brak przerw między słowami co utrudnia aplikacji dodawanie trudnych słów do bazy danych. Ponadto, wprowadzenie funkcji społecznościowych, takich jak fora dyskusyjne czy możliwość współpracy z innymi użytkownikami, mogłoby zwiększyć zaangażowanie i motywację do nauki.

Dalszy rozwój aplikacji powinien obejmować rozbudowę jej funkcjonalności oraz poszerzenie zakresu dostępnych możliwości. W pierwszej kolejności wskazane jest wprowadzenie integracji z popularnymi platformami streamingowymi, co pozwoli na korzystanie z różnorodnych materiałów wideo bez konieczności dodatkowej konfiguracji.

Panel nauki może zostać wzbogacony o nowe metody wspierające proces edukacyjny, takie jak gry edukacyjne, interaktywne ćwiczenia czy testy wiedzy. Rozszerzenie aplikacji o obsługę większej ilości języków mogłoby zostać zrealizowane poprzez wdrożenie zaawansowanych modeli przetwarzania języka naturalnego (NLP), które umożliwią naukę mniej popularnych języków.

Wprowadzenie funkcji społecznościowych, takich jak fora dyskusyjne, grupy nauki czy możliwość współpracy między użytkownikami, stanowiłoby kolejny krok w rozwoju aplikacji. Takie rozwiązania mogłyby sprzyjać budowaniu motywacji oraz zwiększać zaangażowanie użytkowników w proces nauki.

Rozszerzenie funkcjonalności aplikacji o mechanizmy personalizacji, takie jak algorytmy rekomendujące treści dostosowane do poziomu i zainteresowań użytkownika, mogłoby znacząco wpłynąć na efektywność nauczania. Dodanie audio do treści w procesie nauki mogłoby stanowić dodatkowy element wspierający naukę, umożliwiając użytkownikom lepsze zrozumienie wymowy i intonacji w naturalnym kontekście językowym.

